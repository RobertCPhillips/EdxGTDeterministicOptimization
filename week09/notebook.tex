
% Default to the notebook output style

    


% Inherit from the specified cell style.




    
\documentclass[11pt]{article}

    
    
    \usepackage[T1]{fontenc}
    % Nicer default font (+ math font) than Computer Modern for most use cases
    \usepackage{mathpazo}

    % Basic figure setup, for now with no caption control since it's done
    % automatically by Pandoc (which extracts ![](path) syntax from Markdown).
    \usepackage{graphicx}
    % We will generate all images so they have a width \maxwidth. This means
    % that they will get their normal width if they fit onto the page, but
    % are scaled down if they would overflow the margins.
    \makeatletter
    \def\maxwidth{\ifdim\Gin@nat@width>\linewidth\linewidth
    \else\Gin@nat@width\fi}
    \makeatother
    \let\Oldincludegraphics\includegraphics
    % Set max figure width to be 80% of text width, for now hardcoded.
    \renewcommand{\includegraphics}[1]{\Oldincludegraphics[width=.8\maxwidth]{#1}}
    % Ensure that by default, figures have no caption (until we provide a
    % proper Figure object with a Caption API and a way to capture that
    % in the conversion process - todo).
    \usepackage{caption}
    \DeclareCaptionLabelFormat{nolabel}{}
    \captionsetup{labelformat=nolabel}

    \usepackage{adjustbox} % Used to constrain images to a maximum size 
    \usepackage{xcolor} % Allow colors to be defined
    \usepackage{enumerate} % Needed for markdown enumerations to work
    \usepackage{geometry} % Used to adjust the document margins
    \usepackage{amsmath} % Equations
    \usepackage{amssymb} % Equations
    \usepackage{textcomp} % defines textquotesingle
    % Hack from http://tex.stackexchange.com/a/47451/13684:
    \AtBeginDocument{%
        \def\PYZsq{\textquotesingle}% Upright quotes in Pygmentized code
    }
    \usepackage{upquote} % Upright quotes for verbatim code
    \usepackage{eurosym} % defines \euro
    \usepackage[mathletters]{ucs} % Extended unicode (utf-8) support
    \usepackage[utf8x]{inputenc} % Allow utf-8 characters in the tex document
    \usepackage{fancyvrb} % verbatim replacement that allows latex
    \usepackage{grffile} % extends the file name processing of package graphics 
                         % to support a larger range 
    % The hyperref package gives us a pdf with properly built
    % internal navigation ('pdf bookmarks' for the table of contents,
    % internal cross-reference links, web links for URLs, etc.)
    \usepackage{hyperref}
    \usepackage{longtable} % longtable support required by pandoc >1.10
    \usepackage{booktabs}  % table support for pandoc > 1.12.2
    \usepackage[inline]{enumitem} % IRkernel/repr support (it uses the enumerate* environment)
    \usepackage[normalem]{ulem} % ulem is needed to support strikethroughs (\sout)
                                % normalem makes italics be italics, not underlines
    

    
    
    % Colors for the hyperref package
    \definecolor{urlcolor}{rgb}{0,.145,.698}
    \definecolor{linkcolor}{rgb}{.71,0.21,0.01}
    \definecolor{citecolor}{rgb}{.12,.54,.11}

    % ANSI colors
    \definecolor{ansi-black}{HTML}{3E424D}
    \definecolor{ansi-black-intense}{HTML}{282C36}
    \definecolor{ansi-red}{HTML}{E75C58}
    \definecolor{ansi-red-intense}{HTML}{B22B31}
    \definecolor{ansi-green}{HTML}{00A250}
    \definecolor{ansi-green-intense}{HTML}{007427}
    \definecolor{ansi-yellow}{HTML}{DDB62B}
    \definecolor{ansi-yellow-intense}{HTML}{B27D12}
    \definecolor{ansi-blue}{HTML}{208FFB}
    \definecolor{ansi-blue-intense}{HTML}{0065CA}
    \definecolor{ansi-magenta}{HTML}{D160C4}
    \definecolor{ansi-magenta-intense}{HTML}{A03196}
    \definecolor{ansi-cyan}{HTML}{60C6C8}
    \definecolor{ansi-cyan-intense}{HTML}{258F8F}
    \definecolor{ansi-white}{HTML}{C5C1B4}
    \definecolor{ansi-white-intense}{HTML}{A1A6B2}

    % commands and environments needed by pandoc snippets
    % extracted from the output of `pandoc -s`
    \providecommand{\tightlist}{%
      \setlength{\itemsep}{0pt}\setlength{\parskip}{0pt}}
    \DefineVerbatimEnvironment{Highlighting}{Verbatim}{commandchars=\\\{\}}
    % Add ',fontsize=\small' for more characters per line
    \newenvironment{Shaded}{}{}
    \newcommand{\KeywordTok}[1]{\textcolor[rgb]{0.00,0.44,0.13}{\textbf{{#1}}}}
    \newcommand{\DataTypeTok}[1]{\textcolor[rgb]{0.56,0.13,0.00}{{#1}}}
    \newcommand{\DecValTok}[1]{\textcolor[rgb]{0.25,0.63,0.44}{{#1}}}
    \newcommand{\BaseNTok}[1]{\textcolor[rgb]{0.25,0.63,0.44}{{#1}}}
    \newcommand{\FloatTok}[1]{\textcolor[rgb]{0.25,0.63,0.44}{{#1}}}
    \newcommand{\CharTok}[1]{\textcolor[rgb]{0.25,0.44,0.63}{{#1}}}
    \newcommand{\StringTok}[1]{\textcolor[rgb]{0.25,0.44,0.63}{{#1}}}
    \newcommand{\CommentTok}[1]{\textcolor[rgb]{0.38,0.63,0.69}{\textit{{#1}}}}
    \newcommand{\OtherTok}[1]{\textcolor[rgb]{0.00,0.44,0.13}{{#1}}}
    \newcommand{\AlertTok}[1]{\textcolor[rgb]{1.00,0.00,0.00}{\textbf{{#1}}}}
    \newcommand{\FunctionTok}[1]{\textcolor[rgb]{0.02,0.16,0.49}{{#1}}}
    \newcommand{\RegionMarkerTok}[1]{{#1}}
    \newcommand{\ErrorTok}[1]{\textcolor[rgb]{1.00,0.00,0.00}{\textbf{{#1}}}}
    \newcommand{\NormalTok}[1]{{#1}}
    
    % Additional commands for more recent versions of Pandoc
    \newcommand{\ConstantTok}[1]{\textcolor[rgb]{0.53,0.00,0.00}{{#1}}}
    \newcommand{\SpecialCharTok}[1]{\textcolor[rgb]{0.25,0.44,0.63}{{#1}}}
    \newcommand{\VerbatimStringTok}[1]{\textcolor[rgb]{0.25,0.44,0.63}{{#1}}}
    \newcommand{\SpecialStringTok}[1]{\textcolor[rgb]{0.73,0.40,0.53}{{#1}}}
    \newcommand{\ImportTok}[1]{{#1}}
    \newcommand{\DocumentationTok}[1]{\textcolor[rgb]{0.73,0.13,0.13}{\textit{{#1}}}}
    \newcommand{\AnnotationTok}[1]{\textcolor[rgb]{0.38,0.63,0.69}{\textbf{\textit{{#1}}}}}
    \newcommand{\CommentVarTok}[1]{\textcolor[rgb]{0.38,0.63,0.69}{\textbf{\textit{{#1}}}}}
    \newcommand{\VariableTok}[1]{\textcolor[rgb]{0.10,0.09,0.49}{{#1}}}
    \newcommand{\ControlFlowTok}[1]{\textcolor[rgb]{0.00,0.44,0.13}{\textbf{{#1}}}}
    \newcommand{\OperatorTok}[1]{\textcolor[rgb]{0.40,0.40,0.40}{{#1}}}
    \newcommand{\BuiltInTok}[1]{{#1}}
    \newcommand{\ExtensionTok}[1]{{#1}}
    \newcommand{\PreprocessorTok}[1]{\textcolor[rgb]{0.74,0.48,0.00}{{#1}}}
    \newcommand{\AttributeTok}[1]{\textcolor[rgb]{0.49,0.56,0.16}{{#1}}}
    \newcommand{\InformationTok}[1]{\textcolor[rgb]{0.38,0.63,0.69}{\textbf{\textit{{#1}}}}}
    \newcommand{\WarningTok}[1]{\textcolor[rgb]{0.38,0.63,0.69}{\textbf{\textit{{#1}}}}}
    
    
    % Define a nice break command that doesn't care if a line doesn't already
    % exist.
    \def\br{\hspace*{\fill} \\* }
    % Math Jax compatability definitions
    \def\gt{>}
    \def\lt{<}
    % Document parameters
    \title{hw9}
    
    
    

    % Pygments definitions
    
\makeatletter
\def\PY@reset{\let\PY@it=\relax \let\PY@bf=\relax%
    \let\PY@ul=\relax \let\PY@tc=\relax%
    \let\PY@bc=\relax \let\PY@ff=\relax}
\def\PY@tok#1{\csname PY@tok@#1\endcsname}
\def\PY@toks#1+{\ifx\relax#1\empty\else%
    \PY@tok{#1}\expandafter\PY@toks\fi}
\def\PY@do#1{\PY@bc{\PY@tc{\PY@ul{%
    \PY@it{\PY@bf{\PY@ff{#1}}}}}}}
\def\PY#1#2{\PY@reset\PY@toks#1+\relax+\PY@do{#2}}

\expandafter\def\csname PY@tok@w\endcsname{\def\PY@tc##1{\textcolor[rgb]{0.73,0.73,0.73}{##1}}}
\expandafter\def\csname PY@tok@c\endcsname{\let\PY@it=\textit\def\PY@tc##1{\textcolor[rgb]{0.25,0.50,0.50}{##1}}}
\expandafter\def\csname PY@tok@cp\endcsname{\def\PY@tc##1{\textcolor[rgb]{0.74,0.48,0.00}{##1}}}
\expandafter\def\csname PY@tok@k\endcsname{\let\PY@bf=\textbf\def\PY@tc##1{\textcolor[rgb]{0.00,0.50,0.00}{##1}}}
\expandafter\def\csname PY@tok@kp\endcsname{\def\PY@tc##1{\textcolor[rgb]{0.00,0.50,0.00}{##1}}}
\expandafter\def\csname PY@tok@kt\endcsname{\def\PY@tc##1{\textcolor[rgb]{0.69,0.00,0.25}{##1}}}
\expandafter\def\csname PY@tok@o\endcsname{\def\PY@tc##1{\textcolor[rgb]{0.40,0.40,0.40}{##1}}}
\expandafter\def\csname PY@tok@ow\endcsname{\let\PY@bf=\textbf\def\PY@tc##1{\textcolor[rgb]{0.67,0.13,1.00}{##1}}}
\expandafter\def\csname PY@tok@nb\endcsname{\def\PY@tc##1{\textcolor[rgb]{0.00,0.50,0.00}{##1}}}
\expandafter\def\csname PY@tok@nf\endcsname{\def\PY@tc##1{\textcolor[rgb]{0.00,0.00,1.00}{##1}}}
\expandafter\def\csname PY@tok@nc\endcsname{\let\PY@bf=\textbf\def\PY@tc##1{\textcolor[rgb]{0.00,0.00,1.00}{##1}}}
\expandafter\def\csname PY@tok@nn\endcsname{\let\PY@bf=\textbf\def\PY@tc##1{\textcolor[rgb]{0.00,0.00,1.00}{##1}}}
\expandafter\def\csname PY@tok@ne\endcsname{\let\PY@bf=\textbf\def\PY@tc##1{\textcolor[rgb]{0.82,0.25,0.23}{##1}}}
\expandafter\def\csname PY@tok@nv\endcsname{\def\PY@tc##1{\textcolor[rgb]{0.10,0.09,0.49}{##1}}}
\expandafter\def\csname PY@tok@no\endcsname{\def\PY@tc##1{\textcolor[rgb]{0.53,0.00,0.00}{##1}}}
\expandafter\def\csname PY@tok@nl\endcsname{\def\PY@tc##1{\textcolor[rgb]{0.63,0.63,0.00}{##1}}}
\expandafter\def\csname PY@tok@ni\endcsname{\let\PY@bf=\textbf\def\PY@tc##1{\textcolor[rgb]{0.60,0.60,0.60}{##1}}}
\expandafter\def\csname PY@tok@na\endcsname{\def\PY@tc##1{\textcolor[rgb]{0.49,0.56,0.16}{##1}}}
\expandafter\def\csname PY@tok@nt\endcsname{\let\PY@bf=\textbf\def\PY@tc##1{\textcolor[rgb]{0.00,0.50,0.00}{##1}}}
\expandafter\def\csname PY@tok@nd\endcsname{\def\PY@tc##1{\textcolor[rgb]{0.67,0.13,1.00}{##1}}}
\expandafter\def\csname PY@tok@s\endcsname{\def\PY@tc##1{\textcolor[rgb]{0.73,0.13,0.13}{##1}}}
\expandafter\def\csname PY@tok@sd\endcsname{\let\PY@it=\textit\def\PY@tc##1{\textcolor[rgb]{0.73,0.13,0.13}{##1}}}
\expandafter\def\csname PY@tok@si\endcsname{\let\PY@bf=\textbf\def\PY@tc##1{\textcolor[rgb]{0.73,0.40,0.53}{##1}}}
\expandafter\def\csname PY@tok@se\endcsname{\let\PY@bf=\textbf\def\PY@tc##1{\textcolor[rgb]{0.73,0.40,0.13}{##1}}}
\expandafter\def\csname PY@tok@sr\endcsname{\def\PY@tc##1{\textcolor[rgb]{0.73,0.40,0.53}{##1}}}
\expandafter\def\csname PY@tok@ss\endcsname{\def\PY@tc##1{\textcolor[rgb]{0.10,0.09,0.49}{##1}}}
\expandafter\def\csname PY@tok@sx\endcsname{\def\PY@tc##1{\textcolor[rgb]{0.00,0.50,0.00}{##1}}}
\expandafter\def\csname PY@tok@m\endcsname{\def\PY@tc##1{\textcolor[rgb]{0.40,0.40,0.40}{##1}}}
\expandafter\def\csname PY@tok@gh\endcsname{\let\PY@bf=\textbf\def\PY@tc##1{\textcolor[rgb]{0.00,0.00,0.50}{##1}}}
\expandafter\def\csname PY@tok@gu\endcsname{\let\PY@bf=\textbf\def\PY@tc##1{\textcolor[rgb]{0.50,0.00,0.50}{##1}}}
\expandafter\def\csname PY@tok@gd\endcsname{\def\PY@tc##1{\textcolor[rgb]{0.63,0.00,0.00}{##1}}}
\expandafter\def\csname PY@tok@gi\endcsname{\def\PY@tc##1{\textcolor[rgb]{0.00,0.63,0.00}{##1}}}
\expandafter\def\csname PY@tok@gr\endcsname{\def\PY@tc##1{\textcolor[rgb]{1.00,0.00,0.00}{##1}}}
\expandafter\def\csname PY@tok@ge\endcsname{\let\PY@it=\textit}
\expandafter\def\csname PY@tok@gs\endcsname{\let\PY@bf=\textbf}
\expandafter\def\csname PY@tok@gp\endcsname{\let\PY@bf=\textbf\def\PY@tc##1{\textcolor[rgb]{0.00,0.00,0.50}{##1}}}
\expandafter\def\csname PY@tok@go\endcsname{\def\PY@tc##1{\textcolor[rgb]{0.53,0.53,0.53}{##1}}}
\expandafter\def\csname PY@tok@gt\endcsname{\def\PY@tc##1{\textcolor[rgb]{0.00,0.27,0.87}{##1}}}
\expandafter\def\csname PY@tok@err\endcsname{\def\PY@bc##1{\setlength{\fboxsep}{0pt}\fcolorbox[rgb]{1.00,0.00,0.00}{1,1,1}{\strut ##1}}}
\expandafter\def\csname PY@tok@kc\endcsname{\let\PY@bf=\textbf\def\PY@tc##1{\textcolor[rgb]{0.00,0.50,0.00}{##1}}}
\expandafter\def\csname PY@tok@kd\endcsname{\let\PY@bf=\textbf\def\PY@tc##1{\textcolor[rgb]{0.00,0.50,0.00}{##1}}}
\expandafter\def\csname PY@tok@kn\endcsname{\let\PY@bf=\textbf\def\PY@tc##1{\textcolor[rgb]{0.00,0.50,0.00}{##1}}}
\expandafter\def\csname PY@tok@kr\endcsname{\let\PY@bf=\textbf\def\PY@tc##1{\textcolor[rgb]{0.00,0.50,0.00}{##1}}}
\expandafter\def\csname PY@tok@bp\endcsname{\def\PY@tc##1{\textcolor[rgb]{0.00,0.50,0.00}{##1}}}
\expandafter\def\csname PY@tok@fm\endcsname{\def\PY@tc##1{\textcolor[rgb]{0.00,0.00,1.00}{##1}}}
\expandafter\def\csname PY@tok@vc\endcsname{\def\PY@tc##1{\textcolor[rgb]{0.10,0.09,0.49}{##1}}}
\expandafter\def\csname PY@tok@vg\endcsname{\def\PY@tc##1{\textcolor[rgb]{0.10,0.09,0.49}{##1}}}
\expandafter\def\csname PY@tok@vi\endcsname{\def\PY@tc##1{\textcolor[rgb]{0.10,0.09,0.49}{##1}}}
\expandafter\def\csname PY@tok@vm\endcsname{\def\PY@tc##1{\textcolor[rgb]{0.10,0.09,0.49}{##1}}}
\expandafter\def\csname PY@tok@sa\endcsname{\def\PY@tc##1{\textcolor[rgb]{0.73,0.13,0.13}{##1}}}
\expandafter\def\csname PY@tok@sb\endcsname{\def\PY@tc##1{\textcolor[rgb]{0.73,0.13,0.13}{##1}}}
\expandafter\def\csname PY@tok@sc\endcsname{\def\PY@tc##1{\textcolor[rgb]{0.73,0.13,0.13}{##1}}}
\expandafter\def\csname PY@tok@dl\endcsname{\def\PY@tc##1{\textcolor[rgb]{0.73,0.13,0.13}{##1}}}
\expandafter\def\csname PY@tok@s2\endcsname{\def\PY@tc##1{\textcolor[rgb]{0.73,0.13,0.13}{##1}}}
\expandafter\def\csname PY@tok@sh\endcsname{\def\PY@tc##1{\textcolor[rgb]{0.73,0.13,0.13}{##1}}}
\expandafter\def\csname PY@tok@s1\endcsname{\def\PY@tc##1{\textcolor[rgb]{0.73,0.13,0.13}{##1}}}
\expandafter\def\csname PY@tok@mb\endcsname{\def\PY@tc##1{\textcolor[rgb]{0.40,0.40,0.40}{##1}}}
\expandafter\def\csname PY@tok@mf\endcsname{\def\PY@tc##1{\textcolor[rgb]{0.40,0.40,0.40}{##1}}}
\expandafter\def\csname PY@tok@mh\endcsname{\def\PY@tc##1{\textcolor[rgb]{0.40,0.40,0.40}{##1}}}
\expandafter\def\csname PY@tok@mi\endcsname{\def\PY@tc##1{\textcolor[rgb]{0.40,0.40,0.40}{##1}}}
\expandafter\def\csname PY@tok@il\endcsname{\def\PY@tc##1{\textcolor[rgb]{0.40,0.40,0.40}{##1}}}
\expandafter\def\csname PY@tok@mo\endcsname{\def\PY@tc##1{\textcolor[rgb]{0.40,0.40,0.40}{##1}}}
\expandafter\def\csname PY@tok@ch\endcsname{\let\PY@it=\textit\def\PY@tc##1{\textcolor[rgb]{0.25,0.50,0.50}{##1}}}
\expandafter\def\csname PY@tok@cm\endcsname{\let\PY@it=\textit\def\PY@tc##1{\textcolor[rgb]{0.25,0.50,0.50}{##1}}}
\expandafter\def\csname PY@tok@cpf\endcsname{\let\PY@it=\textit\def\PY@tc##1{\textcolor[rgb]{0.25,0.50,0.50}{##1}}}
\expandafter\def\csname PY@tok@c1\endcsname{\let\PY@it=\textit\def\PY@tc##1{\textcolor[rgb]{0.25,0.50,0.50}{##1}}}
\expandafter\def\csname PY@tok@cs\endcsname{\let\PY@it=\textit\def\PY@tc##1{\textcolor[rgb]{0.25,0.50,0.50}{##1}}}

\def\PYZbs{\char`\\}
\def\PYZus{\char`\_}
\def\PYZob{\char`\{}
\def\PYZcb{\char`\}}
\def\PYZca{\char`\^}
\def\PYZam{\char`\&}
\def\PYZlt{\char`\<}
\def\PYZgt{\char`\>}
\def\PYZsh{\char`\#}
\def\PYZpc{\char`\%}
\def\PYZdl{\char`\$}
\def\PYZhy{\char`\-}
\def\PYZsq{\char`\'}
\def\PYZdq{\char`\"}
\def\PYZti{\char`\~}
% for compatibility with earlier versions
\def\PYZat{@}
\def\PYZlb{[}
\def\PYZrb{]}
\makeatother


    % Exact colors from NB
    \definecolor{incolor}{rgb}{0.0, 0.0, 0.5}
    \definecolor{outcolor}{rgb}{0.545, 0.0, 0.0}



    
    % Prevent overflowing lines due to hard-to-break entities
    \sloppy 
    % Setup hyperref package
    \hypersetup{
      breaklinks=true,  % so long urls are correctly broken across lines
      colorlinks=true,
      urlcolor=urlcolor,
      linkcolor=linkcolor,
      citecolor=citecolor,
      }
    % Slightly bigger margins than the latex defaults
    
    \geometry{verbose,tmargin=1in,bmargin=1in,lmargin=1in,rmargin=1in}
    
    

    \begin{document}
    
    
    \maketitle
    
    

    
    \hypertarget{hw-9}{%
\section{HW 9}\label{hw-9}}

    \hypertarget{question-1}{%
\subsection{Question 1}\label{question-1}}

Given the following LP:

\(\begin{align} \text{max } & 2x_1 + x_2 + 4x_3 + 15x_4 \\ \text{s.t. } & 4x_1 + x_2 + 2x_3 + 3x_4 \le 700 \\ & 4x_1 + 2x_2 + x_3 + 5x_4 \le 700 \\ & x_1, x_2, x_3, x_4 \ge 0 \end{align}\)

\hypertarget{form-the-dual-minimization-problem.}{%
\paragraph{1. Form the dual minimization
problem.}\label{form-the-dual-minimization-problem.}}

The given LP is a maximization therefore the dual will be a minimization
problem. This leads to a dual that is minimizing an upper bound for the
given LP. Furthermore, by Strong Duality, the solution to the dual LP is
also as solution to the primal (given) LP.

The dual of the given LP is:

\(\begin{align} \text{min } & 700y_1 + 700y_2 \\ \text{s.t. } & 4y_1 + 4y_2 \ge 2 \\ & y_1 + 2y_2 \ge 1 \\ & 2y_1 + y_2 \ge 4 \\ & 3y_1 + 5y_2 \ge 15 \\ & y_1, y_2 \ge 0 \end{align}\)

\hypertarget{draw-the-feasible-region-of-your-dual-problem.}{%
\paragraph{2. Draw the feasible region of your dual
problem.}\label{draw-the-feasible-region-of-your-dual-problem.}}

    \begin{Verbatim}[commandchars=\\\{\}]
{\color{incolor}In [{\color{incolor}18}]:} \PY{c+c1}{\PYZsh{}plot the solution using matplotlib}
         \PY{k+kn}{import} \PY{n+nn}{matplotlib}\PY{n+nn}{.}\PY{n+nn}{pyplot} \PY{k}{as} \PY{n+nn}{plt}
         \PY{k+kn}{import} \PY{n+nn}{numpy} \PY{k}{as} \PY{n+nn}{np}
         
         \PY{c+c1}{\PYZsh{} x\PYZhy{}values for our plot}
         \PY{n}{xmax} \PY{o}{=} \PY{l+m+mf}{5.0}
         \PY{n}{ymax} \PY{o}{=} \PY{l+m+mf}{4.5}
         \PY{n}{x} \PY{o}{=} \PY{n}{np}\PY{o}{.}\PY{n}{arange}\PY{p}{(}\PY{l+m+mi}{0}\PY{p}{,} \PY{n}{xmax}\PY{p}{,} \PY{l+m+mf}{0.1}\PY{p}{)}
         
         \PY{c+c1}{\PYZsh{} the constraints to plot}
         \PY{n}{y1} \PY{o}{=} \PY{l+m+mf}{2.}  \PY{o}{/} \PY{l+m+mf}{4.} \PY{o}{\PYZhy{}} \PY{l+m+mf}{4.}\PY{o}{*}\PY{n}{x} \PY{o}{/} \PY{l+m+mf}{4.}
         \PY{n}{y2} \PY{o}{=} \PY{l+m+mf}{1.}  \PY{o}{/} \PY{l+m+mf}{2.} \PY{o}{\PYZhy{}} \PY{l+m+mf}{1.}\PY{o}{*}\PY{n}{x} \PY{o}{/} \PY{l+m+mf}{2.}
         \PY{n}{y3} \PY{o}{=} \PY{l+m+mf}{4.}  \PY{o}{/} \PY{l+m+mf}{1.} \PY{o}{\PYZhy{}} \PY{l+m+mf}{2.}\PY{o}{*}\PY{n}{x} \PY{o}{/} \PY{l+m+mf}{1.}
         \PY{n}{y4} \PY{o}{=} \PY{l+m+mf}{15.} \PY{o}{/} \PY{l+m+mf}{5.} \PY{o}{\PYZhy{}} \PY{l+m+mf}{3.}\PY{o}{*}\PY{n}{x} \PY{o}{/} \PY{l+m+mf}{5.}
         
         \PY{c+c1}{\PYZsh{} plot the constraints}
         \PY{n}{plt}\PY{o}{.}\PY{n}{xlim}\PY{p}{(}\PY{l+m+mi}{0}\PY{p}{,} \PY{n}{xmax}\PY{p}{)}
         \PY{n}{plt}\PY{o}{.}\PY{n}{ylim}\PY{p}{(}\PY{l+m+mi}{0}\PY{p}{,} \PY{n}{ymax}\PY{p}{)}
         \PY{n}{plt}\PY{o}{.}\PY{n}{plot}\PY{p}{(}\PY{n}{x}\PY{p}{,} \PY{n}{y1}\PY{p}{,} \PY{n}{x}\PY{p}{,} \PY{n}{y2}\PY{p}{,} \PY{n}{x}\PY{p}{,} \PY{n}{y3}\PY{p}{,} \PY{n}{x}\PY{p}{,} \PY{n}{y4}\PY{p}{,} \PY{n}{label}\PY{o}{=}\PY{l+s+s1}{\PYZsq{}}\PY{l+s+s1}{Feasible Region}\PY{l+s+s1}{\PYZsq{}}\PY{p}{)}
         \PY{n}{plt}\PY{o}{.}\PY{n}{legend}\PY{p}{(}\PY{p}{[}\PY{l+s+sa}{r}\PY{l+s+s1}{\PYZsq{}}\PY{l+s+s1}{\PYZdl{}4y\PYZus{}1 + 4y\PYZus{}2 \PYZgt{}= 2\PYZdl{}}\PY{l+s+s1}{\PYZsq{}}\PY{p}{,} \PY{l+s+sa}{r}\PY{l+s+s1}{\PYZsq{}}\PY{l+s+s1}{\PYZdl{}y\PYZus{}1 + 2y\PYZus{}2 \PYZgt{}= 1\PYZdl{}}\PY{l+s+s1}{\PYZsq{}}\PY{p}{,} \PY{l+s+sa}{r}\PY{l+s+s1}{\PYZsq{}}\PY{l+s+s1}{\PYZdl{}2y\PYZus{}1 + y\PYZus{}2 \PYZgt{}= 4\PYZdl{}}\PY{l+s+s1}{\PYZsq{}}\PY{p}{,} \PY{l+s+sa}{r}\PY{l+s+s1}{\PYZsq{}}\PY{l+s+s1}{\PYZdl{}3y\PYZus{}1 + 5y\PYZus{}2 \PYZgt{}= 15\PYZdl{}}\PY{l+s+s1}{\PYZsq{}}\PY{p}{]}\PY{p}{)}\PY{p}{;}
         \PY{n}{plt}\PY{o}{.}\PY{n}{xlabel}\PY{p}{(}\PY{l+s+sa}{r}\PY{l+s+s1}{\PYZsq{}}\PY{l+s+s1}{\PYZdl{}y\PYZus{}1\PYZdl{}}\PY{l+s+s1}{\PYZsq{}}\PY{p}{)}\PY{p}{;}
         \PY{n}{plt}\PY{o}{.}\PY{n}{ylabel}\PY{p}{(}\PY{l+s+sa}{r}\PY{l+s+s1}{\PYZsq{}}\PY{l+s+s1}{\PYZdl{}y\PYZus{}2\PYZdl{}}\PY{l+s+s1}{\PYZsq{}}\PY{p}{)}\PY{p}{;}
         
         \PY{c+c1}{\PYZsh{} fill in the feasable region (using a polygon)}
         \PY{n}{xp} \PY{o}{=} \PY{p}{[}\PY{l+m+mf}{0.}\PY{p}{,} \PY{l+m+mf}{0.}\PY{p}{,} \PY{l+m+mf}{5.}\PY{p}{,} \PY{l+m+mf}{5.}\PY{p}{,} \PY{l+m+mf}{5.}\PY{o}{/}\PY{l+m+mf}{7.}\PY{p}{]}
         \PY{n}{yp} \PY{o}{=} \PY{p}{[}\PY{l+m+mf}{4.}\PY{p}{,} \PY{l+m+mf}{4.5}\PY{p}{,} \PY{l+m+mf}{4.5}\PY{p}{,} \PY{l+m+mf}{0.}\PY{p}{,} \PY{l+m+mf}{18.}\PY{o}{/}\PY{l+m+mf}{7.}\PY{p}{]}
         \PY{n}{plt}\PY{o}{.}\PY{n}{fill}\PY{p}{(}\PY{n}{xp} \PY{p}{,}\PY{n}{yp}\PY{p}{,} \PY{n}{hatch}\PY{o}{=}\PY{l+s+s1}{\PYZsq{}}\PY{l+s+se}{\PYZbs{}\PYZbs{}}\PY{l+s+s1}{\PYZsq{}}\PY{p}{)}\PY{p}{;}
         
         \PY{n}{plt}\PY{o}{.}\PY{n}{plot}\PY{p}{(}\PY{p}{[}\PY{l+m+mf}{0.}\PY{p}{,} \PY{l+m+mf}{5.}\PY{p}{]}\PY{p}{,} \PY{p}{[}\PY{l+m+mf}{4.}\PY{p}{,} \PY{l+m+mf}{0.}\PY{p}{]}\PY{p}{,} \PY{l+s+s1}{\PYZsq{}}\PY{l+s+s1}{or}\PY{l+s+s1}{\PYZsq{}}\PY{p}{,} \PY{n}{markersize}\PY{o}{=}\PY{l+m+mi}{12}\PY{p}{,} \PY{n}{color}\PY{o}{=}\PY{l+s+s1}{\PYZsq{}}\PY{l+s+s1}{red}\PY{l+s+s1}{\PYZsq{}}\PY{p}{)}\PY{p}{;}
         \PY{n}{plt}\PY{o}{.}\PY{n}{plot}\PY{p}{(}\PY{p}{[}\PY{l+m+mf}{5.}\PY{o}{/}\PY{l+m+mf}{7.}\PY{p}{]}\PY{p}{,} \PY{p}{[}\PY{l+m+mf}{18.}\PY{o}{/}\PY{l+m+mf}{7.}\PY{p}{]}\PY{p}{,} \PY{l+s+s1}{\PYZsq{}}\PY{l+s+s1}{or}\PY{l+s+s1}{\PYZsq{}}\PY{p}{,} \PY{n}{markersize}\PY{o}{=}\PY{l+m+mi}{12}\PY{p}{,} \PY{n}{color}\PY{o}{=}\PY{l+s+s1}{\PYZsq{}}\PY{l+s+s1}{green}\PY{l+s+s1}{\PYZsq{}}\PY{p}{)}\PY{p}{;}
\end{Verbatim}


    \begin{center}
    \adjustimage{max size={0.9\linewidth}{0.9\paperheight}}{output_2_0.png}
    \end{center}
    { \hspace*{\fill} \\}
    
    \hypertarget{point-out-on-the-graph-the-optimal-solution-of-the-dual-problem.}{%
\paragraph{3. Point out on the graph the optimal solution of the dual
problem.}\label{point-out-on-the-graph-the-optimal-solution-of-the-dual-problem.}}

The optimal solution could be at \((y_1,y_2) = (0,4)\) or at
\((y_1,y_2) = (5,0)\) or at \((y_1, y_2) = (\frac{5}{7},\frac{18}{7})\).
These feasible solutions yield the values \(2800, 3500, 2300\)
respectively. Therefore the optimal value is at
\((y_1, y_2) = (\frac{5}{7},\frac{18}{7})\).

    \hypertarget{what-is-the-optimal-objective-value-of-the-dual-problem}{%
\paragraph{4. What is the optimal objective value of the dual
problem?}\label{what-is-the-optimal-objective-value-of-the-dual-problem}}

The optimal value of the dual problem is \(2300\) which is \(200\) more
than Long John.

    \hypertarget{which-primal-constraints-must-be-active-at-the-primal-optimal-solution}{%
\paragraph{5. Which primal constraints must be active at the primal
optimal
solution?}\label{which-primal-constraints-must-be-active-at-the-primal-optimal-solution}}

Excluding non-negativity constraints, the primal LP has 2 constraints
whereas the the dual LP has 4 constraints. In the dual LP, we can see
that only 2 of the constraints are active in the solution. This
corresponds to the variables \(x_3\) and \(x_4\) in the primal LP.

    \hypertarget{which-primal-variables-of-the-optimal-primal-solution-must-be-zero}{%
\paragraph{6. Which primal variables of the optimal primal solution must
be
zero?}\label{which-primal-variables-of-the-optimal-primal-solution-must-be-zero}}

As explained in part 5, 2 of the dual constraints are not active. This
corresponds to the primal variables \(x_1\) and \(x_2\). Therefore they
have a value of zero.

    \hypertarget{find-out-the-primal-optimal-solution.}{%
\paragraph{7. Find out the primal optimal
solution.}\label{find-out-the-primal-optimal-solution.}}

Given that \(x_1\) and \(x_2\) are zero, the primal LP reduces to the
following simple 2 linear equations:

\(2x_3 + 3x_4 = 700\)

\(x_3 + 5x_4 = 700\)

The solution is \((x_1, x_2, x_3, x_4) = (0, 0, 200, 100)\). This gives
an optimal value of \(2*0 + 1*0 + 4*200 + 15*100 = 2300\). This agrees
with the dual solution as expected.

For fun, we can also solve with cvxpy.

    \begin{Verbatim}[commandchars=\\\{\}]
{\color{incolor}In [{\color{incolor}25}]:} \PY{k+kn}{import} \PY{n+nn}{cvxpy} \PY{k}{as} \PY{n+nn}{cp}
         \PY{k+kn}{import} \PY{n+nn}{numpy} \PY{k}{as} \PY{n+nn}{np}
         
         \PY{c+c1}{\PYZsh{}setup variables and coeffcients}
         \PY{n}{x} \PY{o}{=} \PY{n}{cp}\PY{o}{.}\PY{n}{Variable}\PY{p}{(}\PY{l+m+mi}{4}\PY{p}{,} \PY{l+m+mi}{1}\PY{p}{)}
         \PY{n}{c} \PY{o}{=} \PY{n}{np}\PY{o}{.}\PY{n}{array}\PY{p}{(}\PY{p}{[}\PY{l+m+mf}{2.}\PY{p}{,} \PY{l+m+mf}{1.}\PY{p}{,} \PY{l+m+mf}{4.}\PY{p}{,} \PY{l+m+mf}{15.}\PY{p}{]}\PY{p}{)}
         \PY{n}{A} \PY{o}{=} \PY{n}{np}\PY{o}{.}\PY{n}{array}\PY{p}{(}\PY{p}{[}\PY{p}{[}\PY{l+m+mf}{4.}\PY{p}{,}\PY{l+m+mf}{1.}\PY{p}{,}\PY{l+m+mf}{2.}\PY{p}{,}\PY{l+m+mf}{3.}\PY{p}{]}\PY{p}{,}\PY{p}{[}\PY{l+m+mf}{4.}\PY{p}{,}\PY{l+m+mf}{2.}\PY{p}{,}\PY{l+m+mf}{1.}\PY{p}{,}\PY{l+m+mf}{5.}\PY{p}{]}\PY{p}{]}\PY{p}{)}
         \PY{n}{b} \PY{o}{=} \PY{n}{np}\PY{o}{.}\PY{n}{array}\PY{p}{(}\PY{p}{[}\PY{l+m+mf}{700.}\PY{p}{,} \PY{l+m+mf}{700.}\PY{p}{]}\PY{p}{)}
         
         \PY{c+c1}{\PYZsh{}setup objective and constraints}
         \PY{n}{objective} \PY{o}{=} \PY{n}{cp}\PY{o}{.}\PY{n}{Maximize}\PY{p}{(}\PY{n}{c}\PY{o}{*}\PY{n}{x}\PY{p}{)}
         \PY{n}{constraints} \PY{o}{=} \PY{p}{[}\PY{n}{A}\PY{o}{*}\PY{n}{x} \PY{o}{\PYZlt{}}\PY{o}{=} \PY{n}{b}\PY{p}{,} \PY{n}{x} \PY{o}{\PYZgt{}}\PY{o}{=} \PY{l+m+mf}{0.}\PY{p}{]}
         
         \PY{c+c1}{\PYZsh{} solve}
         \PY{n}{prob} \PY{o}{=} \PY{n}{cp}\PY{o}{.}\PY{n}{Problem}\PY{p}{(}\PY{n}{objective}\PY{p}{,} \PY{n}{constraints}\PY{p}{)}
         \PY{n}{result} \PY{o}{=} \PY{n}{prob}\PY{o}{.}\PY{n}{solve}\PY{p}{(}\PY{p}{)}
         
         \PY{c+c1}{\PYZsh{} display optimal value of variables}
         \PY{n+nb}{print}\PY{p}{(}\PY{l+s+s1}{\PYZsq{}}\PY{l+s+s1}{The solution status is}\PY{l+s+s1}{\PYZsq{}}\PY{p}{,} \PY{n}{prob}\PY{o}{.}\PY{n}{status}\PY{p}{)}
         \PY{n+nb}{print}\PY{p}{(}\PY{l+s+s1}{\PYZsq{}}\PY{l+s+s1}{The optimal value is}\PY{l+s+s1}{\PYZsq{}}\PY{p}{,} \PY{n+nb}{round}\PY{p}{(}\PY{n}{result}\PY{p}{)}\PY{p}{)}
         \PY{n+nb}{print}\PY{p}{(}\PY{l+s+s1}{\PYZsq{}}\PY{l+s+s1}{The optimal [x1, x2, x3, x4] is}\PY{l+s+s1}{\PYZsq{}}\PY{p}{,} \PY{p}{[}\PY{n+nb}{round}\PY{p}{(}\PY{n}{xx}\PY{p}{[}\PY{l+m+mi}{0}\PY{p}{,}\PY{l+m+mi}{0}\PY{p}{]}\PY{p}{,}\PY{l+m+mi}{2}\PY{p}{)} \PY{k}{for} \PY{n}{xx} \PY{o+ow}{in} \PY{n}{x}\PY{o}{.}\PY{n}{value}\PY{p}{]}\PY{p}{)}
\end{Verbatim}


    \begin{Verbatim}[commandchars=\\\{\}]
The solution status is optimal
The optimal value is  2300
The optimal [x1, x2, x3, x4] is  [0.0, 0.0, 200.0, 100.0]

    \end{Verbatim}

    \hypertarget{question-2}{%
\subsection{Question 2}\label{question-2}}

\hypertarget{show-that-when-the-budget-for-uncertainty-is-very-large-the-budgeted-uncertainty-is-reduced-to-the-box-uncertainty-set.}{%
\paragraph{1. Show that when the budget for uncertainty is very large,
the budgeted uncertainty is reduced to the box uncertainty
set.}\label{show-that-when-the-budget-for-uncertainty-is-very-large-the-budgeted-uncertainty-is-reduced-to-the-box-uncertainty-set.}}

The constraint with a budget of \(n\) has the following form.

\(\sum_{i=0}^{n} \frac{|d_i - \bar{d_i}|}{\hat{d_i}} \le n\)

The LHS has n terms:

\(\frac{|d_1 - \bar{d_1}|}{\hat{d_1}} + ... + \frac{|d_n - \bar{d_n}|}{\hat{d_n}} \le n\)

Each term above is positive, therefore the above will be true if each
term is \(\le 1\). In other words, the following must be true for each
term:

\(\frac{|d_1 - \bar{d_1}|}{\hat{d_1}} \le 1 \Leftrightarrow |d_1 - \bar{d_1}| \le \hat{d_1} \text{which gives 2 constraints:  } d_1 - \bar{d_1} \le \hat{d_1} \text{ and } d_1 - \bar{d_1} \ge -\hat{d_1}\)

Those 2 constraints can be rearranged as
\(\bar{d_1} + \hat{d_1} \ge d_1 \text{ and } \bar{d_1} - \hat{d_1} \le d_1\)

This is equivalent to the box uncertainty constraint
\(d_i \in [\bar{d_i} - \hat{d_i}, \bar{d_i} + \hat{d_i}]\)

\hypertarget{reformulate-the-given-budget-as-deterministic-linear-constraints.}{%
\paragraph{2. Reformulate the given budget as deterministic linear
constraints.}\label{reformulate-the-given-budget-as-deterministic-linear-constraints.}}

We are given \(\sum_{i=0}^n x_id_i \le g \forall d_i \in D_{budget}\)

This is the same as
\(\sum_{i=0}^n x_i\frac{w_i}{\hat{d_i}} \le g \forall d_i \in D_{budget}\)

This is reformulated as:

\$ \sum\_\{i=0\}\^{}n x\_i\frac{|d_i - \bar{d_i}|}{\hat{d_i}} \le g
\textbackslash{} -d\_i \le -\bar\{d\_i\} + \hat{d_i} \textbackslash{}
d\_i \le \bar\{d\_i\} + \hat{d_i} \textbackslash{} -w\_i + d\_i
\le \bar\{d\_i\} \textbackslash{} -w\_i - d\_i \le -\bar\{d\_i\} \$


    % Add a bibliography block to the postdoc
    
    
    
    \end{document}
