
% Default to the notebook output style

    


% Inherit from the specified cell style.




    
\documentclass[11pt]{article}

    
    
    \usepackage[T1]{fontenc}
    % Nicer default font (+ math font) than Computer Modern for most use cases
    \usepackage{mathpazo}

    % Basic figure setup, for now with no caption control since it's done
    % automatically by Pandoc (which extracts ![](path) syntax from Markdown).
    \usepackage{graphicx}
    % We will generate all images so they have a width \maxwidth. This means
    % that they will get their normal width if they fit onto the page, but
    % are scaled down if they would overflow the margins.
    \makeatletter
    \def\maxwidth{\ifdim\Gin@nat@width>\linewidth\linewidth
    \else\Gin@nat@width\fi}
    \makeatother
    \let\Oldincludegraphics\includegraphics
    % Set max figure width to be 80% of text width, for now hardcoded.
    \renewcommand{\includegraphics}[1]{\Oldincludegraphics[width=.8\maxwidth]{#1}}
    % Ensure that by default, figures have no caption (until we provide a
    % proper Figure object with a Caption API and a way to capture that
    % in the conversion process - todo).
    \usepackage{caption}
    \DeclareCaptionLabelFormat{nolabel}{}
    \captionsetup{labelformat=nolabel}

    \usepackage{adjustbox} % Used to constrain images to a maximum size 
    \usepackage{xcolor} % Allow colors to be defined
    \usepackage{enumerate} % Needed for markdown enumerations to work
    \usepackage{geometry} % Used to adjust the document margins
    \usepackage{amsmath} % Equations
    \usepackage{amssymb} % Equations
    \usepackage{textcomp} % defines textquotesingle
    % Hack from http://tex.stackexchange.com/a/47451/13684:
    \AtBeginDocument{%
        \def\PYZsq{\textquotesingle}% Upright quotes in Pygmentized code
    }
    \usepackage{upquote} % Upright quotes for verbatim code
    \usepackage{eurosym} % defines \euro
    \usepackage[mathletters]{ucs} % Extended unicode (utf-8) support
    \usepackage[utf8x]{inputenc} % Allow utf-8 characters in the tex document
    \usepackage{fancyvrb} % verbatim replacement that allows latex
    \usepackage{grffile} % extends the file name processing of package graphics 
                         % to support a larger range 
    % The hyperref package gives us a pdf with properly built
    % internal navigation ('pdf bookmarks' for the table of contents,
    % internal cross-reference links, web links for URLs, etc.)
    \usepackage{hyperref}
    \usepackage{longtable} % longtable support required by pandoc >1.10
    \usepackage{booktabs}  % table support for pandoc > 1.12.2
    \usepackage[inline]{enumitem} % IRkernel/repr support (it uses the enumerate* environment)
    \usepackage[normalem]{ulem} % ulem is needed to support strikethroughs (\sout)
                                % normalem makes italics be italics, not underlines
    

    
    
    % Colors for the hyperref package
    \definecolor{urlcolor}{rgb}{0,.145,.698}
    \definecolor{linkcolor}{rgb}{.71,0.21,0.01}
    \definecolor{citecolor}{rgb}{.12,.54,.11}

    % ANSI colors
    \definecolor{ansi-black}{HTML}{3E424D}
    \definecolor{ansi-black-intense}{HTML}{282C36}
    \definecolor{ansi-red}{HTML}{E75C58}
    \definecolor{ansi-red-intense}{HTML}{B22B31}
    \definecolor{ansi-green}{HTML}{00A250}
    \definecolor{ansi-green-intense}{HTML}{007427}
    \definecolor{ansi-yellow}{HTML}{DDB62B}
    \definecolor{ansi-yellow-intense}{HTML}{B27D12}
    \definecolor{ansi-blue}{HTML}{208FFB}
    \definecolor{ansi-blue-intense}{HTML}{0065CA}
    \definecolor{ansi-magenta}{HTML}{D160C4}
    \definecolor{ansi-magenta-intense}{HTML}{A03196}
    \definecolor{ansi-cyan}{HTML}{60C6C8}
    \definecolor{ansi-cyan-intense}{HTML}{258F8F}
    \definecolor{ansi-white}{HTML}{C5C1B4}
    \definecolor{ansi-white-intense}{HTML}{A1A6B2}

    % commands and environments needed by pandoc snippets
    % extracted from the output of `pandoc -s`
    \providecommand{\tightlist}{%
      \setlength{\itemsep}{0pt}\setlength{\parskip}{0pt}}
    \DefineVerbatimEnvironment{Highlighting}{Verbatim}{commandchars=\\\{\}}
    % Add ',fontsize=\small' for more characters per line
    \newenvironment{Shaded}{}{}
    \newcommand{\KeywordTok}[1]{\textcolor[rgb]{0.00,0.44,0.13}{\textbf{{#1}}}}
    \newcommand{\DataTypeTok}[1]{\textcolor[rgb]{0.56,0.13,0.00}{{#1}}}
    \newcommand{\DecValTok}[1]{\textcolor[rgb]{0.25,0.63,0.44}{{#1}}}
    \newcommand{\BaseNTok}[1]{\textcolor[rgb]{0.25,0.63,0.44}{{#1}}}
    \newcommand{\FloatTok}[1]{\textcolor[rgb]{0.25,0.63,0.44}{{#1}}}
    \newcommand{\CharTok}[1]{\textcolor[rgb]{0.25,0.44,0.63}{{#1}}}
    \newcommand{\StringTok}[1]{\textcolor[rgb]{0.25,0.44,0.63}{{#1}}}
    \newcommand{\CommentTok}[1]{\textcolor[rgb]{0.38,0.63,0.69}{\textit{{#1}}}}
    \newcommand{\OtherTok}[1]{\textcolor[rgb]{0.00,0.44,0.13}{{#1}}}
    \newcommand{\AlertTok}[1]{\textcolor[rgb]{1.00,0.00,0.00}{\textbf{{#1}}}}
    \newcommand{\FunctionTok}[1]{\textcolor[rgb]{0.02,0.16,0.49}{{#1}}}
    \newcommand{\RegionMarkerTok}[1]{{#1}}
    \newcommand{\ErrorTok}[1]{\textcolor[rgb]{1.00,0.00,0.00}{\textbf{{#1}}}}
    \newcommand{\NormalTok}[1]{{#1}}
    
    % Additional commands for more recent versions of Pandoc
    \newcommand{\ConstantTok}[1]{\textcolor[rgb]{0.53,0.00,0.00}{{#1}}}
    \newcommand{\SpecialCharTok}[1]{\textcolor[rgb]{0.25,0.44,0.63}{{#1}}}
    \newcommand{\VerbatimStringTok}[1]{\textcolor[rgb]{0.25,0.44,0.63}{{#1}}}
    \newcommand{\SpecialStringTok}[1]{\textcolor[rgb]{0.73,0.40,0.53}{{#1}}}
    \newcommand{\ImportTok}[1]{{#1}}
    \newcommand{\DocumentationTok}[1]{\textcolor[rgb]{0.73,0.13,0.13}{\textit{{#1}}}}
    \newcommand{\AnnotationTok}[1]{\textcolor[rgb]{0.38,0.63,0.69}{\textbf{\textit{{#1}}}}}
    \newcommand{\CommentVarTok}[1]{\textcolor[rgb]{0.38,0.63,0.69}{\textbf{\textit{{#1}}}}}
    \newcommand{\VariableTok}[1]{\textcolor[rgb]{0.10,0.09,0.49}{{#1}}}
    \newcommand{\ControlFlowTok}[1]{\textcolor[rgb]{0.00,0.44,0.13}{\textbf{{#1}}}}
    \newcommand{\OperatorTok}[1]{\textcolor[rgb]{0.40,0.40,0.40}{{#1}}}
    \newcommand{\BuiltInTok}[1]{{#1}}
    \newcommand{\ExtensionTok}[1]{{#1}}
    \newcommand{\PreprocessorTok}[1]{\textcolor[rgb]{0.74,0.48,0.00}{{#1}}}
    \newcommand{\AttributeTok}[1]{\textcolor[rgb]{0.49,0.56,0.16}{{#1}}}
    \newcommand{\InformationTok}[1]{\textcolor[rgb]{0.38,0.63,0.69}{\textbf{\textit{{#1}}}}}
    \newcommand{\WarningTok}[1]{\textcolor[rgb]{0.38,0.63,0.69}{\textbf{\textit{{#1}}}}}
    
    
    % Define a nice break command that doesn't care if a line doesn't already
    % exist.
    \def\br{\hspace*{\fill} \\* }
    % Math Jax compatability definitions
    \def\gt{>}
    \def\lt{<}
    % Document parameters
    \title{hw5}
    
    
    

    % Pygments definitions
    
\makeatletter
\def\PY@reset{\let\PY@it=\relax \let\PY@bf=\relax%
    \let\PY@ul=\relax \let\PY@tc=\relax%
    \let\PY@bc=\relax \let\PY@ff=\relax}
\def\PY@tok#1{\csname PY@tok@#1\endcsname}
\def\PY@toks#1+{\ifx\relax#1\empty\else%
    \PY@tok{#1}\expandafter\PY@toks\fi}
\def\PY@do#1{\PY@bc{\PY@tc{\PY@ul{%
    \PY@it{\PY@bf{\PY@ff{#1}}}}}}}
\def\PY#1#2{\PY@reset\PY@toks#1+\relax+\PY@do{#2}}

\expandafter\def\csname PY@tok@w\endcsname{\def\PY@tc##1{\textcolor[rgb]{0.73,0.73,0.73}{##1}}}
\expandafter\def\csname PY@tok@c\endcsname{\let\PY@it=\textit\def\PY@tc##1{\textcolor[rgb]{0.25,0.50,0.50}{##1}}}
\expandafter\def\csname PY@tok@cp\endcsname{\def\PY@tc##1{\textcolor[rgb]{0.74,0.48,0.00}{##1}}}
\expandafter\def\csname PY@tok@k\endcsname{\let\PY@bf=\textbf\def\PY@tc##1{\textcolor[rgb]{0.00,0.50,0.00}{##1}}}
\expandafter\def\csname PY@tok@kp\endcsname{\def\PY@tc##1{\textcolor[rgb]{0.00,0.50,0.00}{##1}}}
\expandafter\def\csname PY@tok@kt\endcsname{\def\PY@tc##1{\textcolor[rgb]{0.69,0.00,0.25}{##1}}}
\expandafter\def\csname PY@tok@o\endcsname{\def\PY@tc##1{\textcolor[rgb]{0.40,0.40,0.40}{##1}}}
\expandafter\def\csname PY@tok@ow\endcsname{\let\PY@bf=\textbf\def\PY@tc##1{\textcolor[rgb]{0.67,0.13,1.00}{##1}}}
\expandafter\def\csname PY@tok@nb\endcsname{\def\PY@tc##1{\textcolor[rgb]{0.00,0.50,0.00}{##1}}}
\expandafter\def\csname PY@tok@nf\endcsname{\def\PY@tc##1{\textcolor[rgb]{0.00,0.00,1.00}{##1}}}
\expandafter\def\csname PY@tok@nc\endcsname{\let\PY@bf=\textbf\def\PY@tc##1{\textcolor[rgb]{0.00,0.00,1.00}{##1}}}
\expandafter\def\csname PY@tok@nn\endcsname{\let\PY@bf=\textbf\def\PY@tc##1{\textcolor[rgb]{0.00,0.00,1.00}{##1}}}
\expandafter\def\csname PY@tok@ne\endcsname{\let\PY@bf=\textbf\def\PY@tc##1{\textcolor[rgb]{0.82,0.25,0.23}{##1}}}
\expandafter\def\csname PY@tok@nv\endcsname{\def\PY@tc##1{\textcolor[rgb]{0.10,0.09,0.49}{##1}}}
\expandafter\def\csname PY@tok@no\endcsname{\def\PY@tc##1{\textcolor[rgb]{0.53,0.00,0.00}{##1}}}
\expandafter\def\csname PY@tok@nl\endcsname{\def\PY@tc##1{\textcolor[rgb]{0.63,0.63,0.00}{##1}}}
\expandafter\def\csname PY@tok@ni\endcsname{\let\PY@bf=\textbf\def\PY@tc##1{\textcolor[rgb]{0.60,0.60,0.60}{##1}}}
\expandafter\def\csname PY@tok@na\endcsname{\def\PY@tc##1{\textcolor[rgb]{0.49,0.56,0.16}{##1}}}
\expandafter\def\csname PY@tok@nt\endcsname{\let\PY@bf=\textbf\def\PY@tc##1{\textcolor[rgb]{0.00,0.50,0.00}{##1}}}
\expandafter\def\csname PY@tok@nd\endcsname{\def\PY@tc##1{\textcolor[rgb]{0.67,0.13,1.00}{##1}}}
\expandafter\def\csname PY@tok@s\endcsname{\def\PY@tc##1{\textcolor[rgb]{0.73,0.13,0.13}{##1}}}
\expandafter\def\csname PY@tok@sd\endcsname{\let\PY@it=\textit\def\PY@tc##1{\textcolor[rgb]{0.73,0.13,0.13}{##1}}}
\expandafter\def\csname PY@tok@si\endcsname{\let\PY@bf=\textbf\def\PY@tc##1{\textcolor[rgb]{0.73,0.40,0.53}{##1}}}
\expandafter\def\csname PY@tok@se\endcsname{\let\PY@bf=\textbf\def\PY@tc##1{\textcolor[rgb]{0.73,0.40,0.13}{##1}}}
\expandafter\def\csname PY@tok@sr\endcsname{\def\PY@tc##1{\textcolor[rgb]{0.73,0.40,0.53}{##1}}}
\expandafter\def\csname PY@tok@ss\endcsname{\def\PY@tc##1{\textcolor[rgb]{0.10,0.09,0.49}{##1}}}
\expandafter\def\csname PY@tok@sx\endcsname{\def\PY@tc##1{\textcolor[rgb]{0.00,0.50,0.00}{##1}}}
\expandafter\def\csname PY@tok@m\endcsname{\def\PY@tc##1{\textcolor[rgb]{0.40,0.40,0.40}{##1}}}
\expandafter\def\csname PY@tok@gh\endcsname{\let\PY@bf=\textbf\def\PY@tc##1{\textcolor[rgb]{0.00,0.00,0.50}{##1}}}
\expandafter\def\csname PY@tok@gu\endcsname{\let\PY@bf=\textbf\def\PY@tc##1{\textcolor[rgb]{0.50,0.00,0.50}{##1}}}
\expandafter\def\csname PY@tok@gd\endcsname{\def\PY@tc##1{\textcolor[rgb]{0.63,0.00,0.00}{##1}}}
\expandafter\def\csname PY@tok@gi\endcsname{\def\PY@tc##1{\textcolor[rgb]{0.00,0.63,0.00}{##1}}}
\expandafter\def\csname PY@tok@gr\endcsname{\def\PY@tc##1{\textcolor[rgb]{1.00,0.00,0.00}{##1}}}
\expandafter\def\csname PY@tok@ge\endcsname{\let\PY@it=\textit}
\expandafter\def\csname PY@tok@gs\endcsname{\let\PY@bf=\textbf}
\expandafter\def\csname PY@tok@gp\endcsname{\let\PY@bf=\textbf\def\PY@tc##1{\textcolor[rgb]{0.00,0.00,0.50}{##1}}}
\expandafter\def\csname PY@tok@go\endcsname{\def\PY@tc##1{\textcolor[rgb]{0.53,0.53,0.53}{##1}}}
\expandafter\def\csname PY@tok@gt\endcsname{\def\PY@tc##1{\textcolor[rgb]{0.00,0.27,0.87}{##1}}}
\expandafter\def\csname PY@tok@err\endcsname{\def\PY@bc##1{\setlength{\fboxsep}{0pt}\fcolorbox[rgb]{1.00,0.00,0.00}{1,1,1}{\strut ##1}}}
\expandafter\def\csname PY@tok@kc\endcsname{\let\PY@bf=\textbf\def\PY@tc##1{\textcolor[rgb]{0.00,0.50,0.00}{##1}}}
\expandafter\def\csname PY@tok@kd\endcsname{\let\PY@bf=\textbf\def\PY@tc##1{\textcolor[rgb]{0.00,0.50,0.00}{##1}}}
\expandafter\def\csname PY@tok@kn\endcsname{\let\PY@bf=\textbf\def\PY@tc##1{\textcolor[rgb]{0.00,0.50,0.00}{##1}}}
\expandafter\def\csname PY@tok@kr\endcsname{\let\PY@bf=\textbf\def\PY@tc##1{\textcolor[rgb]{0.00,0.50,0.00}{##1}}}
\expandafter\def\csname PY@tok@bp\endcsname{\def\PY@tc##1{\textcolor[rgb]{0.00,0.50,0.00}{##1}}}
\expandafter\def\csname PY@tok@fm\endcsname{\def\PY@tc##1{\textcolor[rgb]{0.00,0.00,1.00}{##1}}}
\expandafter\def\csname PY@tok@vc\endcsname{\def\PY@tc##1{\textcolor[rgb]{0.10,0.09,0.49}{##1}}}
\expandafter\def\csname PY@tok@vg\endcsname{\def\PY@tc##1{\textcolor[rgb]{0.10,0.09,0.49}{##1}}}
\expandafter\def\csname PY@tok@vi\endcsname{\def\PY@tc##1{\textcolor[rgb]{0.10,0.09,0.49}{##1}}}
\expandafter\def\csname PY@tok@vm\endcsname{\def\PY@tc##1{\textcolor[rgb]{0.10,0.09,0.49}{##1}}}
\expandafter\def\csname PY@tok@sa\endcsname{\def\PY@tc##1{\textcolor[rgb]{0.73,0.13,0.13}{##1}}}
\expandafter\def\csname PY@tok@sb\endcsname{\def\PY@tc##1{\textcolor[rgb]{0.73,0.13,0.13}{##1}}}
\expandafter\def\csname PY@tok@sc\endcsname{\def\PY@tc##1{\textcolor[rgb]{0.73,0.13,0.13}{##1}}}
\expandafter\def\csname PY@tok@dl\endcsname{\def\PY@tc##1{\textcolor[rgb]{0.73,0.13,0.13}{##1}}}
\expandafter\def\csname PY@tok@s2\endcsname{\def\PY@tc##1{\textcolor[rgb]{0.73,0.13,0.13}{##1}}}
\expandafter\def\csname PY@tok@sh\endcsname{\def\PY@tc##1{\textcolor[rgb]{0.73,0.13,0.13}{##1}}}
\expandafter\def\csname PY@tok@s1\endcsname{\def\PY@tc##1{\textcolor[rgb]{0.73,0.13,0.13}{##1}}}
\expandafter\def\csname PY@tok@mb\endcsname{\def\PY@tc##1{\textcolor[rgb]{0.40,0.40,0.40}{##1}}}
\expandafter\def\csname PY@tok@mf\endcsname{\def\PY@tc##1{\textcolor[rgb]{0.40,0.40,0.40}{##1}}}
\expandafter\def\csname PY@tok@mh\endcsname{\def\PY@tc##1{\textcolor[rgb]{0.40,0.40,0.40}{##1}}}
\expandafter\def\csname PY@tok@mi\endcsname{\def\PY@tc##1{\textcolor[rgb]{0.40,0.40,0.40}{##1}}}
\expandafter\def\csname PY@tok@il\endcsname{\def\PY@tc##1{\textcolor[rgb]{0.40,0.40,0.40}{##1}}}
\expandafter\def\csname PY@tok@mo\endcsname{\def\PY@tc##1{\textcolor[rgb]{0.40,0.40,0.40}{##1}}}
\expandafter\def\csname PY@tok@ch\endcsname{\let\PY@it=\textit\def\PY@tc##1{\textcolor[rgb]{0.25,0.50,0.50}{##1}}}
\expandafter\def\csname PY@tok@cm\endcsname{\let\PY@it=\textit\def\PY@tc##1{\textcolor[rgb]{0.25,0.50,0.50}{##1}}}
\expandafter\def\csname PY@tok@cpf\endcsname{\let\PY@it=\textit\def\PY@tc##1{\textcolor[rgb]{0.25,0.50,0.50}{##1}}}
\expandafter\def\csname PY@tok@c1\endcsname{\let\PY@it=\textit\def\PY@tc##1{\textcolor[rgb]{0.25,0.50,0.50}{##1}}}
\expandafter\def\csname PY@tok@cs\endcsname{\let\PY@it=\textit\def\PY@tc##1{\textcolor[rgb]{0.25,0.50,0.50}{##1}}}

\def\PYZbs{\char`\\}
\def\PYZus{\char`\_}
\def\PYZob{\char`\{}
\def\PYZcb{\char`\}}
\def\PYZca{\char`\^}
\def\PYZam{\char`\&}
\def\PYZlt{\char`\<}
\def\PYZgt{\char`\>}
\def\PYZsh{\char`\#}
\def\PYZpc{\char`\%}
\def\PYZdl{\char`\$}
\def\PYZhy{\char`\-}
\def\PYZsq{\char`\'}
\def\PYZdq{\char`\"}
\def\PYZti{\char`\~}
% for compatibility with earlier versions
\def\PYZat{@}
\def\PYZlb{[}
\def\PYZrb{]}
\makeatother


    % Exact colors from NB
    \definecolor{incolor}{rgb}{0.0, 0.0, 0.5}
    \definecolor{outcolor}{rgb}{0.545, 0.0, 0.0}



    
    % Prevent overflowing lines due to hard-to-break entities
    \sloppy 
    % Setup hyperref package
    \hypersetup{
      breaklinks=true,  % so long urls are correctly broken across lines
      colorlinks=true,
      urlcolor=urlcolor,
      linkcolor=linkcolor,
      citecolor=citecolor,
      }
    % Slightly bigger margins than the latex defaults
    
    \geometry{verbose,tmargin=1in,bmargin=1in,lmargin=1in,rmargin=1in}
    
    

    \begin{document}
    
    
    \maketitle
    
    

    
    \hypertarget{homework-5}{%
\section{Homework 5}\label{homework-5}}

    \hypertarget{question-1}{%
\subsection{Question 1}\label{question-1}}

Consider the following linear program. Then Answer the following
questions.

\(\begin{aligned} \text{min} & \ 3x_1 + x_2 \\ \text{s.t.} & \ x_1 + 2x_2 & \ge 2 \\  & \ 2x_1 + x_2 & \ge 3 \\  & \ x_1 & \ge 0 \\  & \ x_2 & \ge 0 \end{aligned}\)

    \hypertarget{a-draw-the-feasible-region-of-this-lp-in-x_1-x_2.}{%
\paragraph{\texorpdfstring{(a) Draw the feasible region of this LP in
\((x_1, x_2)\).}{(a) Draw the feasible region of this LP in (x\_1, x\_2).}}\label{a-draw-the-feasible-region-of-this-lp-in-x_1-x_2.}}

    \begin{Verbatim}[commandchars=\\\{\}]
{\color{incolor}In [{\color{incolor}3}]:} \PY{c+c1}{\PYZsh{}plot the solution using matplotlib}
        \PY{k+kn}{import} \PY{n+nn}{matplotlib}\PY{n+nn}{.}\PY{n+nn}{pyplot} \PY{k}{as} \PY{n+nn}{plt}
        \PY{k+kn}{import} \PY{n+nn}{numpy} \PY{k}{as} \PY{n+nn}{np}
        
        \PY{c+c1}{\PYZsh{} x\PYZhy{}values for our plot}
        \PY{n}{xmax} \PY{o}{=} \PY{l+m+mi}{5}
        \PY{n}{ymax} \PY{o}{=} \PY{l+m+mi}{4}
        \PY{n}{x} \PY{o}{=} \PY{n}{np}\PY{o}{.}\PY{n}{arange}\PY{p}{(}\PY{l+m+mi}{0}\PY{p}{,} \PY{n}{xmax}\PY{p}{,} \PY{l+m+mf}{0.1}\PY{p}{)}
        
        \PY{c+c1}{\PYZsh{} the constraints to plot}
        \PY{n}{y1} \PY{o}{=} \PY{l+m+mi}{1} \PY{o}{\PYZhy{}} \PY{n}{x} \PY{o}{/} \PY{l+m+mf}{2.}
        \PY{n}{y2} \PY{o}{=} \PY{l+m+mi}{3} \PY{o}{\PYZhy{}} \PY{l+m+mf}{2.}\PY{o}{*}\PY{n}{x}
        
        \PY{c+c1}{\PYZsh{} plot the constraints}
        \PY{n}{plt}\PY{o}{.}\PY{n}{xlim}\PY{p}{(}\PY{l+m+mi}{0}\PY{p}{,} \PY{n}{xmax}\PY{p}{)}
        \PY{n}{plt}\PY{o}{.}\PY{n}{ylim}\PY{p}{(}\PY{l+m+mi}{0}\PY{p}{,} \PY{n}{ymax}\PY{p}{)}
        \PY{n}{plt}\PY{o}{.}\PY{n}{plot}\PY{p}{(}\PY{n}{x}\PY{p}{,} \PY{n}{y1}\PY{p}{,} \PY{n}{x}\PY{p}{,} \PY{n}{y2}\PY{p}{,} \PY{n}{label}\PY{o}{=}\PY{l+s+s1}{\PYZsq{}}\PY{l+s+s1}{Feasible Region}\PY{l+s+s1}{\PYZsq{}}\PY{p}{)}
        \PY{n}{plt}\PY{o}{.}\PY{n}{legend}\PY{p}{(}\PY{p}{[}\PY{l+s+sa}{r}\PY{l+s+s1}{\PYZsq{}}\PY{l+s+s1}{\PYZdl{}x\PYZus{}1 + 2x\PYZus{}2 \PYZgt{}= 2\PYZdl{}}\PY{l+s+s1}{\PYZsq{}}\PY{p}{,} \PY{l+s+sa}{r}\PY{l+s+s1}{\PYZsq{}}\PY{l+s+s1}{\PYZdl{}2x\PYZus{}1 + x\PYZus{}2 \PYZgt{}= 3\PYZdl{}}\PY{l+s+s1}{\PYZsq{}}\PY{p}{]}\PY{p}{)}\PY{p}{;}
        
        \PY{c+c1}{\PYZsh{} fill in the feasable region (using a polygon)}
        \PY{n}{xp} \PY{o}{=} \PY{p}{[}\PY{l+m+mi}{0}\PY{p}{,} \PY{l+m+mi}{0}\PY{p}{,} \PY{n}{xmax}\PY{p}{,} \PY{n}{xmax}\PY{p}{,} \PY{l+m+mi}{2}\PY{p}{,} \PY{l+m+mf}{4.}\PY{o}{/}\PY{l+m+mf}{3.}\PY{p}{]}
        \PY{n}{yp} \PY{o}{=} \PY{p}{[}\PY{l+m+mi}{3}\PY{p}{,} \PY{n}{ymax}\PY{p}{,} \PY{n}{ymax}\PY{p}{,} \PY{l+m+mi}{0}\PY{p}{,} \PY{l+m+mi}{0}\PY{p}{,} \PY{l+m+mf}{1.}\PY{o}{/}\PY{l+m+mf}{3.}\PY{p}{]}
        \PY{n}{plt}\PY{o}{.}\PY{n}{fill}\PY{p}{(}\PY{n}{xp} \PY{p}{,}\PY{n}{yp}\PY{p}{,} \PY{n}{hatch}\PY{o}{=}\PY{l+s+s1}{\PYZsq{}}\PY{l+s+se}{\PYZbs{}\PYZbs{}}\PY{l+s+s1}{\PYZsq{}}\PY{p}{)}\PY{p}{;}
\end{Verbatim}


    \begin{center}
    \adjustimage{max size={0.9\linewidth}{0.9\paperheight}}{output_3_0.png}
    \end{center}
    { \hspace*{\fill} \\}
    
    \hypertarget{b-find-the-optimal-solution-using-the-picture-of-the-feasible-region.-hint-the-optimal-solution-should-be-one-of-the-corner-points-of-the-feasible-region.}{%
\paragraph{(b) Find the optimal solution using the picture of the
feasible region. Hint: The optimal solution should be one of the corner
points of the feasible
region.}\label{b-find-the-optimal-solution-using-the-picture-of-the-feasible-region.-hint-the-optimal-solution-should-be-one-of-the-corner-points-of-the-feasible-region.}}

The optimal solution may be at \((x_1, x_2)\) of \((0, 3)\),
\((\frac{4}{3}, \frac{1}{3})\), or \((2, 0)\). This yields objective
values of \(3\), \(\frac{13}{3}\), and \(6\) repsectively. Therefore the
optimal objective is at \((0,3)\) with value \(3\).

\hypertarget{c-write-a-cvxpy-code-to-find-the-optimal-solution.}{%
\paragraph{(c) Write a CVXPY code to find the optimal
solution.}\label{c-write-a-cvxpy-code-to-find-the-optimal-solution.}}

    \begin{Verbatim}[commandchars=\\\{\}]
{\color{incolor}In [{\color{incolor}4}]:} \PY{k+kn}{import} \PY{n+nn}{cvxpy} \PY{k}{as} \PY{n+nn}{cp}
        \PY{k+kn}{import} \PY{n+nn}{numpy} \PY{k}{as} \PY{n+nn}{np}
        
        \PY{c+c1}{\PYZsh{}setup variables and coeffcients}
        \PY{n}{x} \PY{o}{=} \PY{n}{cp}\PY{o}{.}\PY{n}{Variable}\PY{p}{(}\PY{l+m+mi}{2}\PY{p}{,} \PY{l+m+mi}{1}\PY{p}{)}
        \PY{n}{c} \PY{o}{=} \PY{n}{np}\PY{o}{.}\PY{n}{array}\PY{p}{(}\PY{p}{[}\PY{l+m+mf}{3.}\PY{p}{,} \PY{l+m+mf}{1.}\PY{p}{]}\PY{p}{)}
        \PY{n}{A} \PY{o}{=} \PY{n}{np}\PY{o}{.}\PY{n}{array}\PY{p}{(}\PY{p}{[}\PY{p}{[}\PY{l+m+mf}{1.}\PY{p}{,}\PY{l+m+mf}{2.}\PY{p}{]}\PY{p}{,}\PY{p}{[}\PY{l+m+mf}{2.}\PY{p}{,}\PY{l+m+mf}{1.}\PY{p}{]}\PY{p}{,}\PY{p}{[}\PY{l+m+mf}{1.}\PY{p}{,}\PY{l+m+mf}{0.}\PY{p}{]}\PY{p}{,}\PY{p}{[}\PY{l+m+mf}{0.}\PY{p}{,}\PY{l+m+mf}{1.}\PY{p}{]}\PY{p}{]}\PY{p}{)}
        \PY{n}{b} \PY{o}{=} \PY{n}{np}\PY{o}{.}\PY{n}{array}\PY{p}{(}\PY{p}{[}\PY{l+m+mf}{2.}\PY{p}{,} \PY{l+m+mf}{3.}\PY{p}{,} \PY{l+m+mf}{0.}\PY{p}{,} \PY{l+m+mf}{0.}\PY{p}{]}\PY{p}{)}
        
        \PY{c+c1}{\PYZsh{}setup objective and constraints}
        \PY{n}{objective} \PY{o}{=} \PY{n}{cp}\PY{o}{.}\PY{n}{Minimize}\PY{p}{(}\PY{n}{c}\PY{o}{*}\PY{n}{x}\PY{p}{)}
        \PY{n}{constraints} \PY{o}{=} \PY{p}{[}\PY{n}{A}\PY{o}{*}\PY{n}{x} \PY{o}{\PYZgt{}}\PY{o}{=} \PY{n}{b}\PY{p}{]}
        
        \PY{c+c1}{\PYZsh{} solve}
        \PY{n}{prob} \PY{o}{=} \PY{n}{cp}\PY{o}{.}\PY{n}{Problem}\PY{p}{(}\PY{n}{objective}\PY{p}{,} \PY{n}{constraints}\PY{p}{)}
        \PY{n}{result} \PY{o}{=} \PY{n}{prob}\PY{o}{.}\PY{n}{solve}\PY{p}{(}\PY{p}{)}
        
        \PY{c+c1}{\PYZsh{} display optimal value of variables}
        \PY{n+nb}{print}\PY{p}{(}\PY{l+s+s1}{\PYZsq{}}\PY{l+s+s1}{The optimal value is }\PY{l+s+s1}{\PYZsq{}}\PY{p}{,} \PY{n+nb}{round}\PY{p}{(}\PY{n}{result}\PY{p}{)}\PY{p}{)}
        \PY{n+nb}{print}\PY{p}{(}\PY{l+s+s1}{\PYZsq{}}\PY{l+s+s1}{The optimal x1, x2 is }\PY{l+s+s1}{\PYZsq{}}\PY{p}{,} \PY{p}{[}\PY{n+nb}{round}\PY{p}{(}\PY{n}{xn}\PY{p}{)} \PY{k}{for} \PY{n}{xn} \PY{o+ow}{in} \PY{n}{x}\PY{o}{.}\PY{n}{value}\PY{p}{]}\PY{p}{)}
\end{Verbatim}


    \begin{Verbatim}[commandchars=\\\{\}]
The optimal value is  3.0
The optimal x1, x2 is  [-0.0, 3.0]

    \end{Verbatim}

    \hypertarget{question-2}{%
\subsection{Question 2}\label{question-2}}

Consider a transportation problem with 4 suppliers and 3 customers. The
supply for each supplier \(s_i\) are given as
\(s_1 = 10, s_2 = 25, s_3 =18, s_4 = 15\). The demand for each consumer
\(d_i\) are given as \(d_1 = 15, d_2 = 20, d_3 = 16\). The unit
transportation cost \(c_{ij}\) between supplier \(i\) and consumer \(j\)
are given as:
\(c_{11} = 2, c_{12} = 3, c_{21} = 4, c_{23} = 5, c_{31} = 2, c_{32} = 3, c_{33} = 4, c_{41} = 5, c_{43} = 3\).

\hypertarget{a-formulate-a-linear-program-to-find-the-minimum-total-transportation-cost-to-satisfy-all-the-demand-the-demand-can-be-exceeded.-write-down-the-lp-with-the-given-data.}{%
\paragraph{(a) Formulate a linear program to find the minimum total
transportation cost to satisfy all the demand (the demand can be
exceeded). Write down the LP with the given
data.}\label{a-formulate-a-linear-program-to-find-the-minimum-total-transportation-cost-to-satisfy-all-the-demand-the-demand-can-be-exceeded.-write-down-the-lp-with-the-given-data.}}

Let \(E\) be the set of edges between the supplier and consumer. Thus
\(E = \{(1,1), (1,2), (2,1), (2,3), (3,1), (3,2), (3,3), (4,1), (4,3)\}\)

Let \(x_{ij}\) be the demand for consumer \(j\) satisfied by supplier
\(i\). This yields the following formulation.

\(\begin{aligned} \text{min} & \ \sum_{E} x_{ij}c_{ij} \ & \forall (i,j) \in E \\ \text{s.t.} & \ \sum_{j=1}^{3} x_{ij} \le s_i & \forall i \in \{1,2,3,4\} \\  & \ \sum_{i=1}^{4} x_{ij} \ge d_j & \forall j \in \{1,2,3\} \\  & \ x_{ij} \ge 0 & \forall (i,j) \in E \end{aligned}\)

\hypertarget{b-write-a-cvxpy-code-to-find-the-optimal-solution-of-the-above-lp.}{%
\paragraph{(b) Write a CVXPY code to find the optimal solution of the
above
LP.}\label{b-write-a-cvxpy-code-to-find-the-optimal-solution-of-the-above-lp.}}

    \begin{Verbatim}[commandchars=\\\{\}]
{\color{incolor}In [{\color{incolor}5}]:} \PY{c+c1}{\PYZsh{} 9 variables for the connected nodes}
        \PY{n}{i} \PY{o}{=} \PY{l+m+mi}{4}
        \PY{n}{j} \PY{o}{=} \PY{l+m+mi}{3}
        \PY{n}{n} \PY{o}{=} \PY{l+m+mi}{9}
        
        \PY{n}{x} \PY{o}{=} \PY{n}{cp}\PY{o}{.}\PY{n}{Variable}\PY{p}{(}\PY{n}{n}\PY{p}{,} \PY{l+m+mi}{1}\PY{p}{)}
        
        \PY{c+c1}{\PYZsh{} cost variables as given}
        \PY{n}{c} \PY{o}{=} \PY{n}{np}\PY{o}{.}\PY{n}{array}\PY{p}{(}\PY{p}{[}\PY{l+m+mf}{2.}\PY{p}{,} \PY{l+m+mf}{3.}\PY{p}{,} \PY{l+m+mf}{4.}\PY{p}{,} \PY{l+m+mf}{5.}\PY{p}{,} \PY{l+m+mf}{2.}\PY{p}{,} \PY{l+m+mf}{3.}\PY{p}{,} \PY{l+m+mf}{4.}\PY{p}{,} \PY{l+m+mf}{5.}\PY{p}{,} \PY{l+m+mf}{3.}\PY{p}{]}\PY{p}{)}
        \PY{n}{bs} \PY{o}{=} \PY{n}{np}\PY{o}{.}\PY{n}{array}\PY{p}{(}\PY{p}{[}\PY{l+m+mf}{10.}\PY{p}{,} \PY{l+m+mf}{25.}\PY{p}{,} \PY{l+m+mf}{18.}\PY{p}{,} \PY{l+m+mf}{15.}\PY{p}{]}\PY{p}{)}
        \PY{n}{bd} \PY{o}{=} \PY{n}{np}\PY{o}{.}\PY{n}{array}\PY{p}{(}\PY{p}{[}\PY{l+m+mf}{15.}\PY{p}{,} \PY{l+m+mf}{20.}\PY{p}{,} \PY{l+m+mf}{16.}\PY{p}{]}\PY{p}{)}
        
        \PY{n}{S} \PY{o}{=} \PY{n}{np}\PY{o}{.}\PY{n}{zeros}\PY{p}{(}\PY{p}{(}\PY{n}{i}\PY{p}{,}\PY{n}{n}\PY{p}{)}\PY{p}{)}
        \PY{n}{S}\PY{p}{[}\PY{l+m+mi}{0}\PY{p}{,}\PY{l+m+mi}{0}\PY{p}{]} \PY{o}{=} \PY{l+m+mf}{1.}\PY{p}{;}\PY{n}{S}\PY{p}{[}\PY{l+m+mi}{0}\PY{p}{,}\PY{l+m+mi}{1}\PY{p}{]} \PY{o}{=} \PY{l+m+mi}{1}\PY{p}{;}
        \PY{n}{S}\PY{p}{[}\PY{l+m+mi}{1}\PY{p}{,}\PY{l+m+mi}{2}\PY{p}{]} \PY{o}{=} \PY{l+m+mf}{1.}\PY{p}{;}\PY{n}{S}\PY{p}{[}\PY{l+m+mi}{1}\PY{p}{,}\PY{l+m+mi}{3}\PY{p}{]} \PY{o}{=} \PY{l+m+mi}{1}\PY{p}{;}
        \PY{n}{S}\PY{p}{[}\PY{l+m+mi}{2}\PY{p}{,}\PY{l+m+mi}{4}\PY{p}{]} \PY{o}{=} \PY{l+m+mf}{1.}\PY{p}{;}\PY{n}{S}\PY{p}{[}\PY{l+m+mi}{2}\PY{p}{,}\PY{l+m+mi}{5}\PY{p}{]} \PY{o}{=} \PY{l+m+mi}{1}\PY{p}{;}\PY{n}{S}\PY{p}{[}\PY{l+m+mi}{2}\PY{p}{,}\PY{l+m+mi}{6}\PY{p}{]} \PY{o}{=} \PY{l+m+mf}{1.}\PY{p}{;}
        \PY{n}{S}\PY{p}{[}\PY{l+m+mi}{3}\PY{p}{,}\PY{l+m+mi}{7}\PY{p}{]} \PY{o}{=} \PY{l+m+mf}{1.}\PY{p}{;}\PY{n}{S}\PY{p}{[}\PY{l+m+mi}{3}\PY{p}{,}\PY{l+m+mi}{8}\PY{p}{]} \PY{o}{=} \PY{l+m+mi}{1}\PY{p}{;}
        
        \PY{n}{D} \PY{o}{=} \PY{n}{np}\PY{o}{.}\PY{n}{zeros}\PY{p}{(}\PY{p}{(}\PY{n}{j}\PY{p}{,}\PY{n}{n}\PY{p}{)}\PY{p}{)}
        \PY{n}{D}\PY{p}{[}\PY{l+m+mi}{0}\PY{p}{,}\PY{l+m+mi}{0}\PY{p}{]} \PY{o}{=} \PY{l+m+mf}{1.}\PY{p}{;}\PY{n}{D}\PY{p}{[}\PY{l+m+mi}{0}\PY{p}{,}\PY{l+m+mi}{2}\PY{p}{]} \PY{o}{=} \PY{l+m+mf}{1.}\PY{p}{;}\PY{n}{D}\PY{p}{[}\PY{l+m+mi}{0}\PY{p}{,}\PY{l+m+mi}{4}\PY{p}{]} \PY{o}{=} \PY{l+m+mf}{1.}\PY{p}{;}\PY{n}{D}\PY{p}{[}\PY{l+m+mi}{0}\PY{p}{,}\PY{l+m+mi}{7}\PY{p}{]} \PY{o}{=} \PY{l+m+mf}{1.}\PY{p}{;}
        \PY{n}{D}\PY{p}{[}\PY{l+m+mi}{1}\PY{p}{,}\PY{l+m+mi}{1}\PY{p}{]} \PY{o}{=} \PY{l+m+mf}{1.}\PY{p}{;}\PY{n}{D}\PY{p}{[}\PY{l+m+mi}{1}\PY{p}{,}\PY{l+m+mi}{5}\PY{p}{]} \PY{o}{=} \PY{l+m+mf}{1.}\PY{p}{;}
        \PY{n}{D}\PY{p}{[}\PY{l+m+mi}{2}\PY{p}{,}\PY{l+m+mi}{3}\PY{p}{]} \PY{o}{=} \PY{l+m+mf}{1.}\PY{p}{;}\PY{n}{D}\PY{p}{[}\PY{l+m+mi}{2}\PY{p}{,}\PY{l+m+mi}{6}\PY{p}{]} \PY{o}{=} \PY{l+m+mf}{1.}\PY{p}{;}\PY{n}{D}\PY{p}{[}\PY{l+m+mi}{2}\PY{p}{,}\PY{l+m+mi}{8}\PY{p}{]} \PY{o}{=} \PY{l+m+mf}{1.}\PY{p}{;}
        
        \PY{c+c1}{\PYZsh{}setup objective and constraints}
        \PY{n}{objective} \PY{o}{=} \PY{n}{cp}\PY{o}{.}\PY{n}{Minimize}\PY{p}{(}\PY{n}{c}\PY{o}{*}\PY{n}{x}\PY{p}{)}
        \PY{n}{constraints} \PY{o}{=} \PY{p}{[}\PY{n}{S}\PY{o}{*}\PY{n}{x} \PY{o}{\PYZlt{}}\PY{o}{=} \PY{n}{bs}\PY{p}{,} \PY{n}{D}\PY{o}{*}\PY{n}{x} \PY{o}{\PYZgt{}}\PY{o}{=} \PY{n}{bd}\PY{p}{,} \PY{n}{x} \PY{o}{\PYZgt{}}\PY{o}{=} \PY{l+m+mi}{0}\PY{p}{]}
        
        \PY{c+c1}{\PYZsh{} solve}
        \PY{n}{prob} \PY{o}{=} \PY{n}{cp}\PY{o}{.}\PY{n}{Problem}\PY{p}{(}\PY{n}{objective}\PY{p}{,} \PY{n}{constraints}\PY{p}{)}
        \PY{n}{result} \PY{o}{=} \PY{n}{prob}\PY{o}{.}\PY{n}{solve}\PY{p}{(}\PY{p}{)}
        
        \PY{c+c1}{\PYZsh{} display optimal value of variables}
        \PY{n+nb}{print}\PY{p}{(}\PY{l+s+s1}{\PYZsq{}}\PY{l+s+s1}{The optimal value is }\PY{l+s+s1}{\PYZsq{}}\PY{p}{,} \PY{n+nb}{round}\PY{p}{(}\PY{n}{result}\PY{p}{)}\PY{p}{)}
        \PY{n+nb}{print}\PY{p}{(}\PY{l+s+s1}{\PYZsq{}}\PY{l+s+s1}{The optimal x is }\PY{l+s+s1}{\PYZsq{}}\PY{p}{,} \PY{p}{[}\PY{n+nb}{round}\PY{p}{(}\PY{n}{xij}\PY{p}{)} \PY{k}{for} \PY{n}{xij} \PY{o+ow}{in} \PY{n}{x}\PY{o}{.}\PY{n}{value}\PY{p}{]}\PY{p}{)}
\end{Verbatim}


    \begin{Verbatim}[commandchars=\\\{\}]
The optimal value is  154.0
The optimal x is  [3.0, 7.0, 7.0, 1.0, 5.0, 13.0, 0.0, -0.0, 15.0]

    \end{Verbatim}

    \hypertarget{c-modify-your-code-so-that-every-demand-is-satisfied-exactly-i.e.cannot-be-exceeded.}{%
\paragraph{(c) Modify your code so that every demand is satisfied
exactly, i.e.~cannot be
exceeded.}\label{c-modify-your-code-so-that-every-demand-is-satisfied-exactly-i.e.cannot-be-exceeded.}}

    \begin{Verbatim}[commandchars=\\\{\}]
{\color{incolor}In [{\color{incolor}6}]:} \PY{c+c1}{\PYZsh{} modify constraint to ==}
        \PY{n}{constraints} \PY{o}{=} \PY{p}{[}\PY{n}{S}\PY{o}{*}\PY{n}{x} \PY{o}{\PYZlt{}}\PY{o}{=} \PY{n}{bs}\PY{p}{,} \PY{n}{D}\PY{o}{*}\PY{n}{x} \PY{o}{==} \PY{n}{bd}\PY{p}{,} \PY{n}{x} \PY{o}{\PYZgt{}}\PY{o}{=} \PY{l+m+mi}{0}\PY{p}{]}
        
        \PY{c+c1}{\PYZsh{} solve}
        \PY{n}{prob} \PY{o}{=} \PY{n}{cp}\PY{o}{.}\PY{n}{Problem}\PY{p}{(}\PY{n}{objective}\PY{p}{,} \PY{n}{constraints}\PY{p}{)}
        \PY{n}{result} \PY{o}{=} \PY{n}{prob}\PY{o}{.}\PY{n}{solve}\PY{p}{(}\PY{p}{)}
        
        \PY{c+c1}{\PYZsh{} display optimal value of variables}
        \PY{n+nb}{print}\PY{p}{(}\PY{l+s+s1}{\PYZsq{}}\PY{l+s+s1}{The optimal value is }\PY{l+s+s1}{\PYZsq{}}\PY{p}{,} \PY{n+nb}{round}\PY{p}{(}\PY{n}{result}\PY{p}{)}\PY{p}{)}
        \PY{n+nb}{print}\PY{p}{(}\PY{l+s+s1}{\PYZsq{}}\PY{l+s+s1}{The optimal x is }\PY{l+s+s1}{\PYZsq{}}\PY{p}{,} \PY{p}{[}\PY{n+nb}{round}\PY{p}{(}\PY{n}{xij}\PY{p}{)} \PY{k}{for} \PY{n}{xij} \PY{o+ow}{in} \PY{n}{x}\PY{o}{.}\PY{n}{value}\PY{p}{]}\PY{p}{)}
\end{Verbatim}


    \begin{Verbatim}[commandchars=\\\{\}]
The optimal value is  154.0
The optimal x is  [1.0, 9.0, 7.0, 1.0, 7.0, 11.0, 0.0, 0.0, 15.0]

    \end{Verbatim}

    The previous result where the demand constraint was changed to an
equality yields a different solution but with the same objective value.

    \hypertarget{question-3}{%
\subsection{Question 3}\label{question-3}}

Consider the following electric power network taken from a real-world
electric power system. Electricity generators are located at nodes 1, 3,
and 5. Electricity loads are located at nodes 2, 4, and 6.

Denote the amount of generation of generator \(i\) as \(p_i\) and the
demand of consumer \(j\) as \(d_j\).

The demand is given as \(d_1 = 90\), \(d_2 = 100\), \(d_3 = 125\). The
range of generation lower and upper bounds are given as
\(p_1^{min} = 10\), \(p_1^{max} = 250\), \(p_2^{min} = 10\),
\(p_2^{max} = 300\), \(p_3^{min} = 10\), \(p_3^{max} = 270\).

The flow limit on lines are given as \(f_{12}^{max} = 50\),
\(f_{23}^{max} = 60\), \(f_{34}^{max} = 90\), \(f_{45}^{max} = 50\),
\(f_{56}^{max} = 120\), \(f_{61}^{max} = 100\).

The line parameters are given as \(B_{12} = 11.6\), \(B_{23} = 5.9\),
\(B_{34} = 13.7\), \(B_{45} = 9.8\), \(B_{56} = 5.6\),
\(B_{61} = 10.5\).

The unit generation costs are given as \(c_1 = 5\), \(c_2 = 2\),
\(c_3 = 3\).

\hypertarget{a-formulate-the-power-system-scheduling-problem-using-the-model-discussed-in-lecture-2.}{%
\paragraph{(a) Formulate the power system scheduling problem using the
model discussed in Lecture
2.}\label{a-formulate-the-power-system-scheduling-problem-using-the-model-discussed-in-lecture-2.}}

\(\text{let} \ \mathbf{E} = \{(1,2),(2,3),(3,4),(4,5),(5,6),(6,1)\} \\ \begin{aligned} \text{min} & \ \sum_{i=1}^{3} p_ic_i \\ \text{s.t.} & \ p_i^{min} \le p_i \le p_i^{max} \ & \forall i \in \{1,2,3\} \\  & \ -f_{ij}^{max} \le f_{ij} \le f_{ij}^{max} \ & \forall (i,j) \in \mathbf{E} \\  & f_{ij} = B_{ij}(\theta_i - \theta_j) \ & \forall (i,j) \in \mathbf{E} \\  & f_{12} - f_{61} = p_1 \\  & -f_{23} + f_{34} = p_2 \\  & -f_{45} + f_{56} = p_3 \\  & f_{12} - f_{23} = d_1 \\  & f_{34} - f_{45} = d_2 \\  & -f_{61} + f_{56} = d_3 \\ \end{aligned}\)

    \hypertarget{b-implement-and-solve-the-model-using-cvxpy.-write-down-the-optimal-solution.}{%
\paragraph{(b) Implement and solve the model using CVXPY. Write down the
optimal
solution.}\label{b-implement-and-solve-the-model-using-cvxpy.-write-down-the-optimal-solution.}}

    \begin{Verbatim}[commandchars=\\\{\}]
{\color{incolor}In [{\color{incolor}71}]:} \PY{c+c1}{\PYZsh{} 6 variables for 6 node potentials (thetas)}
         \PY{n}{n} \PY{o}{=} \PY{l+m+mi}{6}
         
         \PY{n}{x} \PY{o}{=} \PY{n}{cp}\PY{o}{.}\PY{n}{Variable}\PY{p}{(}\PY{n}{n}\PY{p}{,} \PY{l+m+mi}{1}\PY{p}{)} \PY{c+c1}{\PYZsh{}theta for each node}
         
         \PY{c+c1}{\PYZsh{} variables as given}
         \PY{n}{c} \PY{o}{=} \PY{n}{np}\PY{o}{.}\PY{n}{array}\PY{p}{(}\PY{p}{[}\PY{l+m+mf}{5.}\PY{p}{,} \PY{l+m+mf}{2.}\PY{p}{,} \PY{l+m+mf}{3.}\PY{p}{]}\PY{p}{)}
         \PY{n}{B} \PY{o}{=} \PY{n}{np}\PY{o}{.}\PY{n}{array}\PY{p}{(}\PY{p}{[}\PY{l+m+mf}{11.6}\PY{p}{,} \PY{l+m+mf}{5.9}\PY{p}{,} \PY{l+m+mf}{13.7}\PY{p}{,} \PY{l+m+mf}{9.8}\PY{p}{,} \PY{l+m+mf}{5.6}\PY{p}{,} \PY{l+m+mf}{10.5}\PY{p}{]}\PY{p}{)}
         \PY{n}{pmin} \PY{o}{=} \PY{n}{np}\PY{o}{.}\PY{n}{array}\PY{p}{(}\PY{p}{[}\PY{l+m+mf}{10.}\PY{p}{,} \PY{l+m+mf}{10.}\PY{p}{,} \PY{l+m+mf}{10.}\PY{p}{]}\PY{p}{)}
         \PY{n}{pmax} \PY{o}{=} \PY{n}{np}\PY{o}{.}\PY{n}{array}\PY{p}{(}\PY{p}{[}\PY{l+m+mf}{250.}\PY{p}{,} \PY{l+m+mf}{300.}\PY{p}{,} \PY{l+m+mf}{270.}\PY{p}{]}\PY{p}{)}
         \PY{n}{fmax} \PY{o}{=} \PY{n}{np}\PY{o}{.}\PY{n}{array}\PY{p}{(}\PY{p}{[}\PY{l+m+mf}{50.}\PY{p}{,} \PY{l+m+mf}{60.}\PY{p}{,} \PY{l+m+mf}{90.}\PY{p}{,} \PY{l+m+mf}{50.}\PY{p}{,} \PY{l+m+mf}{120.}\PY{p}{,} \PY{l+m+mf}{100.}\PY{p}{]}\PY{p}{)}
         \PY{n}{d} \PY{o}{=} \PY{n}{np}\PY{o}{.}\PY{n}{array}\PY{p}{(}\PY{p}{[}\PY{l+m+mf}{90.}\PY{p}{,} \PY{l+m+mf}{100.}\PY{p}{,} \PY{l+m+mf}{125.}\PY{p}{]}\PY{p}{)}
         
         \PY{c+c1}{\PYZsh{}generator node flow conservation (as a function of theta and B)}
         \PY{n}{P} \PY{o}{=} \PY{n}{np}\PY{o}{.}\PY{n}{array}\PY{p}{(}\PY{p}{[}\PY{p}{[}\PY{n}{B}\PY{p}{[}\PY{l+m+mi}{0}\PY{p}{]}\PY{o}{+}\PY{n}{B}\PY{p}{[}\PY{l+m+mi}{5}\PY{p}{]}\PY{p}{,} \PY{o}{\PYZhy{}}\PY{n}{B}\PY{p}{[}\PY{l+m+mi}{0}\PY{p}{]}\PY{p}{,} \PY{l+m+mf}{0.}\PY{p}{,} \PY{l+m+mf}{0.}\PY{p}{,} \PY{l+m+mf}{0.}\PY{p}{,} \PY{o}{\PYZhy{}}\PY{n}{B}\PY{p}{[}\PY{l+m+mi}{5}\PY{p}{]}\PY{p}{]}\PY{p}{,}
                       \PY{p}{[}\PY{l+m+mf}{0.}\PY{p}{,} \PY{o}{\PYZhy{}}\PY{n}{B}\PY{p}{[}\PY{l+m+mi}{1}\PY{p}{]}\PY{p}{,} \PY{n}{B}\PY{p}{[}\PY{l+m+mi}{1}\PY{p}{]}\PY{o}{+}\PY{n}{B}\PY{p}{[}\PY{l+m+mi}{2}\PY{p}{]}\PY{p}{,} \PY{o}{\PYZhy{}}\PY{n}{B}\PY{p}{[}\PY{l+m+mi}{2}\PY{p}{]}\PY{p}{,} \PY{l+m+mf}{0.}\PY{p}{,} \PY{l+m+mf}{0.}\PY{p}{]}\PY{p}{,}
                       \PY{p}{[}\PY{l+m+mf}{0.}\PY{p}{,} \PY{l+m+mf}{0.}\PY{p}{,} \PY{l+m+mf}{0.}\PY{p}{,} \PY{o}{\PYZhy{}}\PY{n}{B}\PY{p}{[}\PY{l+m+mi}{3}\PY{p}{]}\PY{p}{,} \PY{n}{B}\PY{p}{[}\PY{l+m+mi}{3}\PY{p}{]}\PY{o}{+}\PY{n}{B}\PY{p}{[}\PY{l+m+mi}{4}\PY{p}{]}\PY{p}{,} \PY{o}{\PYZhy{}}\PY{n}{B}\PY{p}{[}\PY{l+m+mi}{4}\PY{p}{]}\PY{p}{]}\PY{p}{]}\PY{p}{)}
         
         \PY{c+c1}{\PYZsh{}demand node flow conservation (as a function of theta and B)}
         \PY{n}{D} \PY{o}{=} \PY{n}{np}\PY{o}{.}\PY{n}{array}\PY{p}{(}\PY{p}{[}\PY{p}{[}\PY{n}{B}\PY{p}{[}\PY{l+m+mi}{0}\PY{p}{]}\PY{p}{,} \PY{o}{\PYZhy{}}\PY{n}{B}\PY{p}{[}\PY{l+m+mi}{0}\PY{p}{]}\PY{o}{\PYZhy{}}\PY{n}{B}\PY{p}{[}\PY{l+m+mi}{1}\PY{p}{]}\PY{p}{,} \PY{n}{B}\PY{p}{[}\PY{l+m+mi}{1}\PY{p}{]}\PY{p}{,} \PY{l+m+mf}{0.}\PY{p}{,} \PY{l+m+mf}{0.}\PY{p}{,} \PY{l+m+mf}{0.}\PY{p}{]}\PY{p}{,}
                       \PY{p}{[}\PY{l+m+mf}{0.}\PY{p}{,} \PY{l+m+mf}{0.}\PY{p}{,} \PY{n}{B}\PY{p}{[}\PY{l+m+mi}{2}\PY{p}{]}\PY{p}{,} \PY{o}{\PYZhy{}}\PY{n}{B}\PY{p}{[}\PY{l+m+mi}{2}\PY{p}{]}\PY{o}{\PYZhy{}}\PY{n}{B}\PY{p}{[}\PY{l+m+mi}{3}\PY{p}{]}\PY{p}{,} \PY{n}{B}\PY{p}{[}\PY{l+m+mi}{3}\PY{p}{]}\PY{p}{,} \PY{l+m+mf}{0.}\PY{p}{]}\PY{p}{,}
                       \PY{p}{[}\PY{n}{B}\PY{p}{[}\PY{l+m+mi}{5}\PY{p}{]}\PY{p}{,} \PY{l+m+mf}{0.}\PY{p}{,} \PY{l+m+mf}{0.}\PY{p}{,} \PY{l+m+mf}{0.}\PY{p}{,} \PY{n}{B}\PY{p}{[}\PY{l+m+mi}{4}\PY{p}{]}\PY{p}{,} \PY{o}{\PYZhy{}}\PY{n}{B}\PY{p}{[}\PY{l+m+mi}{4}\PY{p}{]}\PY{o}{\PYZhy{}}\PY{n}{B}\PY{p}{[}\PY{l+m+mi}{5}\PY{p}{]}\PY{p}{]}\PY{p}{]}\PY{p}{)}
         
         \PY{c+c1}{\PYZsh{}flow limits (as a function of theta and B \PYZhy{} f\PYZus{}ij = B\PYZus{}ij(theta\PYZus{}i\PYZhy{}theta\PYZus{}j))}
         \PY{n}{F} \PY{o}{=} \PY{n}{np}\PY{o}{.}\PY{n}{array}\PY{p}{(}\PY{p}{[}\PY{p}{[}\PY{n}{B}\PY{p}{[}\PY{l+m+mi}{0}\PY{p}{]}\PY{p}{,} \PY{o}{\PYZhy{}}\PY{n}{B}\PY{p}{[}\PY{l+m+mi}{0}\PY{p}{]}\PY{p}{,} \PY{l+m+mi}{0}\PY{p}{,} \PY{l+m+mi}{0}\PY{p}{,} \PY{l+m+mi}{0}\PY{p}{,} \PY{l+m+mi}{0}\PY{p}{]}\PY{p}{,}
                       \PY{p}{[}\PY{l+m+mi}{0}\PY{p}{,} \PY{n}{B}\PY{p}{[}\PY{l+m+mi}{1}\PY{p}{]}\PY{p}{,} \PY{o}{\PYZhy{}}\PY{n}{B}\PY{p}{[}\PY{l+m+mi}{1}\PY{p}{]}\PY{p}{,} \PY{l+m+mi}{0}\PY{p}{,} \PY{l+m+mi}{0}\PY{p}{,} \PY{l+m+mi}{0}\PY{p}{]}\PY{p}{,}
                       \PY{p}{[}\PY{l+m+mi}{0}\PY{p}{,} \PY{l+m+mi}{0}\PY{p}{,} \PY{n}{B}\PY{p}{[}\PY{l+m+mi}{2}\PY{p}{]}\PY{p}{,} \PY{o}{\PYZhy{}}\PY{n}{B}\PY{p}{[}\PY{l+m+mi}{2}\PY{p}{]}\PY{p}{,} \PY{l+m+mi}{0}\PY{p}{,} \PY{l+m+mi}{0}\PY{p}{]}\PY{p}{,}
                       \PY{p}{[}\PY{l+m+mi}{0}\PY{p}{,} \PY{l+m+mi}{0}\PY{p}{,} \PY{l+m+mi}{0}\PY{p}{,} \PY{n}{B}\PY{p}{[}\PY{l+m+mi}{3}\PY{p}{]}\PY{p}{,} \PY{o}{\PYZhy{}}\PY{n}{B}\PY{p}{[}\PY{l+m+mi}{3}\PY{p}{]}\PY{p}{,} \PY{l+m+mi}{0}\PY{p}{]}\PY{p}{,}
                       \PY{p}{[}\PY{l+m+mi}{0}\PY{p}{,} \PY{l+m+mi}{0}\PY{p}{,} \PY{l+m+mi}{0}\PY{p}{,} \PY{l+m+mi}{0}\PY{p}{,} \PY{n}{B}\PY{p}{[}\PY{l+m+mi}{4}\PY{p}{]}\PY{p}{,} \PY{o}{\PYZhy{}}\PY{n}{B}\PY{p}{[}\PY{l+m+mi}{4}\PY{p}{]}\PY{p}{]}\PY{p}{,}
                       \PY{p}{[}\PY{o}{\PYZhy{}}\PY{n}{B}\PY{p}{[}\PY{l+m+mi}{5}\PY{p}{]}\PY{p}{,} \PY{l+m+mi}{0}\PY{p}{,} \PY{l+m+mi}{0}\PY{p}{,} \PY{l+m+mi}{0}\PY{p}{,} \PY{l+m+mi}{0}\PY{p}{,} \PY{n}{B}\PY{p}{[}\PY{l+m+mi}{5}\PY{p}{]}\PY{p}{]}\PY{p}{]}\PY{p}{)}
         
         \PY{c+c1}{\PYZsh{}setup objective and constraints}
         \PY{n}{objective} \PY{o}{=} \PY{n}{cp}\PY{o}{.}\PY{n}{Minimize}\PY{p}{(}\PY{n}{P}\PY{o}{*}\PY{n}{x}\PY{o}{*}\PY{n}{c}\PY{p}{)}
         \PY{n}{constraints} \PY{o}{=} \PY{p}{[}\PY{n}{P}\PY{o}{*}\PY{n}{x} \PY{o}{\PYZlt{}}\PY{o}{=} \PY{n}{pmax}\PY{p}{,} \PY{n}{P}\PY{o}{*}\PY{n}{x} \PY{o}{\PYZgt{}}\PY{o}{=} \PY{n}{pmin}\PY{p}{,} \PY{n}{D}\PY{o}{*}\PY{n}{x} \PY{o}{==} \PY{n}{d}\PY{p}{,} \PY{n}{F}\PY{o}{*}\PY{n}{x} \PY{o}{\PYZgt{}}\PY{o}{=} \PY{o}{\PYZhy{}}\PY{n}{fmax}\PY{p}{,} \PY{n}{F}\PY{o}{*}\PY{n}{x} \PY{o}{\PYZlt{}}\PY{o}{=} \PY{n}{fmax}\PY{p}{]}
         
         \PY{c+c1}{\PYZsh{} solve}
         \PY{n}{prob} \PY{o}{=} \PY{n}{cp}\PY{o}{.}\PY{n}{Problem}\PY{p}{(}\PY{n}{objective}\PY{p}{,} \PY{n}{constraints}\PY{p}{)}
         \PY{n}{result} \PY{o}{=} \PY{n}{prob}\PY{o}{.}\PY{n}{solve}\PY{p}{(}\PY{p}{)}
         
         \PY{c+c1}{\PYZsh{} display optimal value of variables}
         \PY{n+nb}{print}\PY{p}{(}\PY{l+s+s1}{\PYZsq{}}\PY{l+s+s1}{The solver status is}\PY{l+s+s1}{\PYZsq{}}\PY{p}{,} \PY{n}{prob}\PY{o}{.}\PY{n}{status}\PY{p}{)}
         \PY{n+nb}{print}\PY{p}{(}\PY{l+s+s1}{\PYZsq{}}\PY{l+s+s1}{The optimal value is}\PY{l+s+s1}{\PYZsq{}}\PY{p}{,} \PY{n+nb}{round}\PY{p}{(}\PY{n}{result}\PY{p}{,} \PY{l+m+mi}{2}\PY{p}{)}\PY{p}{)}
         \PY{n+nb}{print}\PY{p}{(}\PY{l+s+s1}{\PYZsq{}}\PY{l+s+s1}{The optimal x (theta\PYZus{}1, ..., theta\PYZus{}6) is}\PY{l+s+s1}{\PYZsq{}}\PY{p}{,} \PY{p}{[}\PY{n+nb}{round}\PY{p}{(}\PY{n}{xi}\PY{p}{,}\PY{l+m+mi}{2}\PY{p}{)} \PY{k}{for} \PY{n}{xi} \PY{o+ow}{in} \PY{n}{x}\PY{o}{.}\PY{n}{value}\PY{p}{]}\PY{p}{)}
         \PY{c+c1}{\PYZsh{}print(P)}
         \PY{c+c1}{\PYZsh{}print(D)}
         \PY{c+c1}{\PYZsh{}print(F)}
         \PY{c+c1}{\PYZsh{}print(prob)}
         \PY{c+c1}{\PYZsh{}print(x.value)}
         \PY{n+nb}{print}\PY{p}{(}\PY{l+s+s1}{\PYZsq{}}\PY{l+s+s1}{The power generation values (p\PYZus{}1, p\PYZus{}2, p\PYZus{}3) are}\PY{l+s+s1}{\PYZsq{}}\PY{p}{,} \PY{p}{[}\PY{n+nb}{round}\PY{p}{(}\PY{n}{p}\PY{p}{,}\PY{l+m+mi}{2}\PY{p}{)} \PY{k}{for} \PY{n}{p} \PY{o+ow}{in} \PY{n}{np}\PY{o}{.}\PY{n}{dot}\PY{p}{(}\PY{n}{P}\PY{p}{,}\PY{n}{x}\PY{o}{.}\PY{n}{value}\PY{p}{)}\PY{p}{]}\PY{p}{)}
\end{Verbatim}


    \begin{Verbatim}[commandchars=\\\{\}]
The solver status is optimal
The optimal value is 992.04
The optimal x (theta\_1, {\ldots}, theta\_6) is  [-2.28, -4.87, 5.3, 1.65, 6.75, -6.9]
The power generation values (p\_1, p\_2, p\_3) are [78.52, 110.0, 126.48]

    \end{Verbatim}

    \hypertarget{c-find-the-electricity-prices-for-demand-at-nodes-2-4-6.-to-do-this-use-the-command-constraints0.dual-value-to-find-the-dual-variable-of-constraints0.-hint-recall-the-electricity-price-at-node-i-is-the-dual-variable-for-the-flow-conservation-constraint-at-node-i.}{%
\paragraph{\texorpdfstring{(c) Find the electricity prices for demand at
nodes \(2\), \(4\), \(6\). To do this, use the command
constraints{[}0{]}.dual value to find the dual variable of
constraints{[}0{]}. Hint: Recall the electricity price at node i is the
dual variable for the flow conservation constraint at node
\(i\).}{(c) Find the electricity prices for demand at nodes 2, 4, 6. To do this, use the command constraints{[}0{]}.dual value to find the dual variable of constraints{[}0{]}. Hint: Recall the electricity price at node i is the dual variable for the flow conservation constraint at node i.}}\label{c-find-the-electricity-prices-for-demand-at-nodes-2-4-6.-to-do-this-use-the-command-constraints0.dual-value-to-find-the-dual-variable-of-constraints0.-hint-recall-the-electricity-price-at-node-i-is-the-dual-variable-for-the-flow-conservation-constraint-at-node-i.}}

The demand nodes are represented by the constraint D*x == d, which is
the 3rd item in the list of constraint variables.

    \begin{Verbatim}[commandchars=\\\{\}]
{\color{incolor}In [{\color{incolor}66}]:} \PY{n+nb}{print}\PY{p}{(}\PY{n}{constraints}\PY{p}{[}\PY{l+m+mi}{2}\PY{p}{]}\PY{o}{.}\PY{n}{dual\PYZus{}value}\PY{p}{)}
\end{Verbatim}


    \begin{Verbatim}[commandchars=\\\{\}]
[-5.62968516 -2.53316407 -4.30434783]

    \end{Verbatim}


    % Add a bibliography block to the postdoc
    
    
    
    \end{document}
