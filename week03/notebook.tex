
% Default to the notebook output style

    


% Inherit from the specified cell style.




    
\documentclass[11pt]{article}

    
    
    \usepackage[T1]{fontenc}
    % Nicer default font (+ math font) than Computer Modern for most use cases
    \usepackage{mathpazo}

    % Basic figure setup, for now with no caption control since it's done
    % automatically by Pandoc (which extracts ![](path) syntax from Markdown).
    \usepackage{graphicx}
    % We will generate all images so they have a width \maxwidth. This means
    % that they will get their normal width if they fit onto the page, but
    % are scaled down if they would overflow the margins.
    \makeatletter
    \def\maxwidth{\ifdim\Gin@nat@width>\linewidth\linewidth
    \else\Gin@nat@width\fi}
    \makeatother
    \let\Oldincludegraphics\includegraphics
    % Set max figure width to be 80% of text width, for now hardcoded.
    \renewcommand{\includegraphics}[1]{\Oldincludegraphics[width=.8\maxwidth]{#1}}
    % Ensure that by default, figures have no caption (until we provide a
    % proper Figure object with a Caption API and a way to capture that
    % in the conversion process - todo).
    \usepackage{caption}
    \DeclareCaptionLabelFormat{nolabel}{}
    \captionsetup{labelformat=nolabel}

    \usepackage{adjustbox} % Used to constrain images to a maximum size 
    \usepackage{xcolor} % Allow colors to be defined
    \usepackage{enumerate} % Needed for markdown enumerations to work
    \usepackage{geometry} % Used to adjust the document margins
    \usepackage{amsmath} % Equations
    \usepackage{amssymb} % Equations
    \usepackage{textcomp} % defines textquotesingle
    % Hack from http://tex.stackexchange.com/a/47451/13684:
    \AtBeginDocument{%
        \def\PYZsq{\textquotesingle}% Upright quotes in Pygmentized code
    }
    \usepackage{upquote} % Upright quotes for verbatim code
    \usepackage{eurosym} % defines \euro
    \usepackage[mathletters]{ucs} % Extended unicode (utf-8) support
    \usepackage[utf8x]{inputenc} % Allow utf-8 characters in the tex document
    \usepackage{fancyvrb} % verbatim replacement that allows latex
    \usepackage{grffile} % extends the file name processing of package graphics 
                         % to support a larger range 
    % The hyperref package gives us a pdf with properly built
    % internal navigation ('pdf bookmarks' for the table of contents,
    % internal cross-reference links, web links for URLs, etc.)
    \usepackage{hyperref}
    \usepackage{longtable} % longtable support required by pandoc >1.10
    \usepackage{booktabs}  % table support for pandoc > 1.12.2
    \usepackage[inline]{enumitem} % IRkernel/repr support (it uses the enumerate* environment)
    \usepackage[normalem]{ulem} % ulem is needed to support strikethroughs (\sout)
                                % normalem makes italics be italics, not underlines
    

    
    
    % Colors for the hyperref package
    \definecolor{urlcolor}{rgb}{0,.145,.698}
    \definecolor{linkcolor}{rgb}{.71,0.21,0.01}
    \definecolor{citecolor}{rgb}{.12,.54,.11}

    % ANSI colors
    \definecolor{ansi-black}{HTML}{3E424D}
    \definecolor{ansi-black-intense}{HTML}{282C36}
    \definecolor{ansi-red}{HTML}{E75C58}
    \definecolor{ansi-red-intense}{HTML}{B22B31}
    \definecolor{ansi-green}{HTML}{00A250}
    \definecolor{ansi-green-intense}{HTML}{007427}
    \definecolor{ansi-yellow}{HTML}{DDB62B}
    \definecolor{ansi-yellow-intense}{HTML}{B27D12}
    \definecolor{ansi-blue}{HTML}{208FFB}
    \definecolor{ansi-blue-intense}{HTML}{0065CA}
    \definecolor{ansi-magenta}{HTML}{D160C4}
    \definecolor{ansi-magenta-intense}{HTML}{A03196}
    \definecolor{ansi-cyan}{HTML}{60C6C8}
    \definecolor{ansi-cyan-intense}{HTML}{258F8F}
    \definecolor{ansi-white}{HTML}{C5C1B4}
    \definecolor{ansi-white-intense}{HTML}{A1A6B2}

    % commands and environments needed by pandoc snippets
    % extracted from the output of `pandoc -s`
    \providecommand{\tightlist}{%
      \setlength{\itemsep}{0pt}\setlength{\parskip}{0pt}}
    \DefineVerbatimEnvironment{Highlighting}{Verbatim}{commandchars=\\\{\}}
    % Add ',fontsize=\small' for more characters per line
    \newenvironment{Shaded}{}{}
    \newcommand{\KeywordTok}[1]{\textcolor[rgb]{0.00,0.44,0.13}{\textbf{{#1}}}}
    \newcommand{\DataTypeTok}[1]{\textcolor[rgb]{0.56,0.13,0.00}{{#1}}}
    \newcommand{\DecValTok}[1]{\textcolor[rgb]{0.25,0.63,0.44}{{#1}}}
    \newcommand{\BaseNTok}[1]{\textcolor[rgb]{0.25,0.63,0.44}{{#1}}}
    \newcommand{\FloatTok}[1]{\textcolor[rgb]{0.25,0.63,0.44}{{#1}}}
    \newcommand{\CharTok}[1]{\textcolor[rgb]{0.25,0.44,0.63}{{#1}}}
    \newcommand{\StringTok}[1]{\textcolor[rgb]{0.25,0.44,0.63}{{#1}}}
    \newcommand{\CommentTok}[1]{\textcolor[rgb]{0.38,0.63,0.69}{\textit{{#1}}}}
    \newcommand{\OtherTok}[1]{\textcolor[rgb]{0.00,0.44,0.13}{{#1}}}
    \newcommand{\AlertTok}[1]{\textcolor[rgb]{1.00,0.00,0.00}{\textbf{{#1}}}}
    \newcommand{\FunctionTok}[1]{\textcolor[rgb]{0.02,0.16,0.49}{{#1}}}
    \newcommand{\RegionMarkerTok}[1]{{#1}}
    \newcommand{\ErrorTok}[1]{\textcolor[rgb]{1.00,0.00,0.00}{\textbf{{#1}}}}
    \newcommand{\NormalTok}[1]{{#1}}
    
    % Additional commands for more recent versions of Pandoc
    \newcommand{\ConstantTok}[1]{\textcolor[rgb]{0.53,0.00,0.00}{{#1}}}
    \newcommand{\SpecialCharTok}[1]{\textcolor[rgb]{0.25,0.44,0.63}{{#1}}}
    \newcommand{\VerbatimStringTok}[1]{\textcolor[rgb]{0.25,0.44,0.63}{{#1}}}
    \newcommand{\SpecialStringTok}[1]{\textcolor[rgb]{0.73,0.40,0.53}{{#1}}}
    \newcommand{\ImportTok}[1]{{#1}}
    \newcommand{\DocumentationTok}[1]{\textcolor[rgb]{0.73,0.13,0.13}{\textit{{#1}}}}
    \newcommand{\AnnotationTok}[1]{\textcolor[rgb]{0.38,0.63,0.69}{\textbf{\textit{{#1}}}}}
    \newcommand{\CommentVarTok}[1]{\textcolor[rgb]{0.38,0.63,0.69}{\textbf{\textit{{#1}}}}}
    \newcommand{\VariableTok}[1]{\textcolor[rgb]{0.10,0.09,0.49}{{#1}}}
    \newcommand{\ControlFlowTok}[1]{\textcolor[rgb]{0.00,0.44,0.13}{\textbf{{#1}}}}
    \newcommand{\OperatorTok}[1]{\textcolor[rgb]{0.40,0.40,0.40}{{#1}}}
    \newcommand{\BuiltInTok}[1]{{#1}}
    \newcommand{\ExtensionTok}[1]{{#1}}
    \newcommand{\PreprocessorTok}[1]{\textcolor[rgb]{0.74,0.48,0.00}{{#1}}}
    \newcommand{\AttributeTok}[1]{\textcolor[rgb]{0.49,0.56,0.16}{{#1}}}
    \newcommand{\InformationTok}[1]{\textcolor[rgb]{0.38,0.63,0.69}{\textbf{\textit{{#1}}}}}
    \newcommand{\WarningTok}[1]{\textcolor[rgb]{0.38,0.63,0.69}{\textbf{\textit{{#1}}}}}
    
    
    % Define a nice break command that doesn't care if a line doesn't already
    % exist.
    \def\br{\hspace*{\fill} \\* }
    % Math Jax compatability definitions
    \def\gt{>}
    \def\lt{<}
    % Document parameters
    \title{hw3}
    
    
    

    % Pygments definitions
    
\makeatletter
\def\PY@reset{\let\PY@it=\relax \let\PY@bf=\relax%
    \let\PY@ul=\relax \let\PY@tc=\relax%
    \let\PY@bc=\relax \let\PY@ff=\relax}
\def\PY@tok#1{\csname PY@tok@#1\endcsname}
\def\PY@toks#1+{\ifx\relax#1\empty\else%
    \PY@tok{#1}\expandafter\PY@toks\fi}
\def\PY@do#1{\PY@bc{\PY@tc{\PY@ul{%
    \PY@it{\PY@bf{\PY@ff{#1}}}}}}}
\def\PY#1#2{\PY@reset\PY@toks#1+\relax+\PY@do{#2}}

\expandafter\def\csname PY@tok@w\endcsname{\def\PY@tc##1{\textcolor[rgb]{0.73,0.73,0.73}{##1}}}
\expandafter\def\csname PY@tok@c\endcsname{\let\PY@it=\textit\def\PY@tc##1{\textcolor[rgb]{0.25,0.50,0.50}{##1}}}
\expandafter\def\csname PY@tok@cp\endcsname{\def\PY@tc##1{\textcolor[rgb]{0.74,0.48,0.00}{##1}}}
\expandafter\def\csname PY@tok@k\endcsname{\let\PY@bf=\textbf\def\PY@tc##1{\textcolor[rgb]{0.00,0.50,0.00}{##1}}}
\expandafter\def\csname PY@tok@kp\endcsname{\def\PY@tc##1{\textcolor[rgb]{0.00,0.50,0.00}{##1}}}
\expandafter\def\csname PY@tok@kt\endcsname{\def\PY@tc##1{\textcolor[rgb]{0.69,0.00,0.25}{##1}}}
\expandafter\def\csname PY@tok@o\endcsname{\def\PY@tc##1{\textcolor[rgb]{0.40,0.40,0.40}{##1}}}
\expandafter\def\csname PY@tok@ow\endcsname{\let\PY@bf=\textbf\def\PY@tc##1{\textcolor[rgb]{0.67,0.13,1.00}{##1}}}
\expandafter\def\csname PY@tok@nb\endcsname{\def\PY@tc##1{\textcolor[rgb]{0.00,0.50,0.00}{##1}}}
\expandafter\def\csname PY@tok@nf\endcsname{\def\PY@tc##1{\textcolor[rgb]{0.00,0.00,1.00}{##1}}}
\expandafter\def\csname PY@tok@nc\endcsname{\let\PY@bf=\textbf\def\PY@tc##1{\textcolor[rgb]{0.00,0.00,1.00}{##1}}}
\expandafter\def\csname PY@tok@nn\endcsname{\let\PY@bf=\textbf\def\PY@tc##1{\textcolor[rgb]{0.00,0.00,1.00}{##1}}}
\expandafter\def\csname PY@tok@ne\endcsname{\let\PY@bf=\textbf\def\PY@tc##1{\textcolor[rgb]{0.82,0.25,0.23}{##1}}}
\expandafter\def\csname PY@tok@nv\endcsname{\def\PY@tc##1{\textcolor[rgb]{0.10,0.09,0.49}{##1}}}
\expandafter\def\csname PY@tok@no\endcsname{\def\PY@tc##1{\textcolor[rgb]{0.53,0.00,0.00}{##1}}}
\expandafter\def\csname PY@tok@nl\endcsname{\def\PY@tc##1{\textcolor[rgb]{0.63,0.63,0.00}{##1}}}
\expandafter\def\csname PY@tok@ni\endcsname{\let\PY@bf=\textbf\def\PY@tc##1{\textcolor[rgb]{0.60,0.60,0.60}{##1}}}
\expandafter\def\csname PY@tok@na\endcsname{\def\PY@tc##1{\textcolor[rgb]{0.49,0.56,0.16}{##1}}}
\expandafter\def\csname PY@tok@nt\endcsname{\let\PY@bf=\textbf\def\PY@tc##1{\textcolor[rgb]{0.00,0.50,0.00}{##1}}}
\expandafter\def\csname PY@tok@nd\endcsname{\def\PY@tc##1{\textcolor[rgb]{0.67,0.13,1.00}{##1}}}
\expandafter\def\csname PY@tok@s\endcsname{\def\PY@tc##1{\textcolor[rgb]{0.73,0.13,0.13}{##1}}}
\expandafter\def\csname PY@tok@sd\endcsname{\let\PY@it=\textit\def\PY@tc##1{\textcolor[rgb]{0.73,0.13,0.13}{##1}}}
\expandafter\def\csname PY@tok@si\endcsname{\let\PY@bf=\textbf\def\PY@tc##1{\textcolor[rgb]{0.73,0.40,0.53}{##1}}}
\expandafter\def\csname PY@tok@se\endcsname{\let\PY@bf=\textbf\def\PY@tc##1{\textcolor[rgb]{0.73,0.40,0.13}{##1}}}
\expandafter\def\csname PY@tok@sr\endcsname{\def\PY@tc##1{\textcolor[rgb]{0.73,0.40,0.53}{##1}}}
\expandafter\def\csname PY@tok@ss\endcsname{\def\PY@tc##1{\textcolor[rgb]{0.10,0.09,0.49}{##1}}}
\expandafter\def\csname PY@tok@sx\endcsname{\def\PY@tc##1{\textcolor[rgb]{0.00,0.50,0.00}{##1}}}
\expandafter\def\csname PY@tok@m\endcsname{\def\PY@tc##1{\textcolor[rgb]{0.40,0.40,0.40}{##1}}}
\expandafter\def\csname PY@tok@gh\endcsname{\let\PY@bf=\textbf\def\PY@tc##1{\textcolor[rgb]{0.00,0.00,0.50}{##1}}}
\expandafter\def\csname PY@tok@gu\endcsname{\let\PY@bf=\textbf\def\PY@tc##1{\textcolor[rgb]{0.50,0.00,0.50}{##1}}}
\expandafter\def\csname PY@tok@gd\endcsname{\def\PY@tc##1{\textcolor[rgb]{0.63,0.00,0.00}{##1}}}
\expandafter\def\csname PY@tok@gi\endcsname{\def\PY@tc##1{\textcolor[rgb]{0.00,0.63,0.00}{##1}}}
\expandafter\def\csname PY@tok@gr\endcsname{\def\PY@tc##1{\textcolor[rgb]{1.00,0.00,0.00}{##1}}}
\expandafter\def\csname PY@tok@ge\endcsname{\let\PY@it=\textit}
\expandafter\def\csname PY@tok@gs\endcsname{\let\PY@bf=\textbf}
\expandafter\def\csname PY@tok@gp\endcsname{\let\PY@bf=\textbf\def\PY@tc##1{\textcolor[rgb]{0.00,0.00,0.50}{##1}}}
\expandafter\def\csname PY@tok@go\endcsname{\def\PY@tc##1{\textcolor[rgb]{0.53,0.53,0.53}{##1}}}
\expandafter\def\csname PY@tok@gt\endcsname{\def\PY@tc##1{\textcolor[rgb]{0.00,0.27,0.87}{##1}}}
\expandafter\def\csname PY@tok@err\endcsname{\def\PY@bc##1{\setlength{\fboxsep}{0pt}\fcolorbox[rgb]{1.00,0.00,0.00}{1,1,1}{\strut ##1}}}
\expandafter\def\csname PY@tok@kc\endcsname{\let\PY@bf=\textbf\def\PY@tc##1{\textcolor[rgb]{0.00,0.50,0.00}{##1}}}
\expandafter\def\csname PY@tok@kd\endcsname{\let\PY@bf=\textbf\def\PY@tc##1{\textcolor[rgb]{0.00,0.50,0.00}{##1}}}
\expandafter\def\csname PY@tok@kn\endcsname{\let\PY@bf=\textbf\def\PY@tc##1{\textcolor[rgb]{0.00,0.50,0.00}{##1}}}
\expandafter\def\csname PY@tok@kr\endcsname{\let\PY@bf=\textbf\def\PY@tc##1{\textcolor[rgb]{0.00,0.50,0.00}{##1}}}
\expandafter\def\csname PY@tok@bp\endcsname{\def\PY@tc##1{\textcolor[rgb]{0.00,0.50,0.00}{##1}}}
\expandafter\def\csname PY@tok@fm\endcsname{\def\PY@tc##1{\textcolor[rgb]{0.00,0.00,1.00}{##1}}}
\expandafter\def\csname PY@tok@vc\endcsname{\def\PY@tc##1{\textcolor[rgb]{0.10,0.09,0.49}{##1}}}
\expandafter\def\csname PY@tok@vg\endcsname{\def\PY@tc##1{\textcolor[rgb]{0.10,0.09,0.49}{##1}}}
\expandafter\def\csname PY@tok@vi\endcsname{\def\PY@tc##1{\textcolor[rgb]{0.10,0.09,0.49}{##1}}}
\expandafter\def\csname PY@tok@vm\endcsname{\def\PY@tc##1{\textcolor[rgb]{0.10,0.09,0.49}{##1}}}
\expandafter\def\csname PY@tok@sa\endcsname{\def\PY@tc##1{\textcolor[rgb]{0.73,0.13,0.13}{##1}}}
\expandafter\def\csname PY@tok@sb\endcsname{\def\PY@tc##1{\textcolor[rgb]{0.73,0.13,0.13}{##1}}}
\expandafter\def\csname PY@tok@sc\endcsname{\def\PY@tc##1{\textcolor[rgb]{0.73,0.13,0.13}{##1}}}
\expandafter\def\csname PY@tok@dl\endcsname{\def\PY@tc##1{\textcolor[rgb]{0.73,0.13,0.13}{##1}}}
\expandafter\def\csname PY@tok@s2\endcsname{\def\PY@tc##1{\textcolor[rgb]{0.73,0.13,0.13}{##1}}}
\expandafter\def\csname PY@tok@sh\endcsname{\def\PY@tc##1{\textcolor[rgb]{0.73,0.13,0.13}{##1}}}
\expandafter\def\csname PY@tok@s1\endcsname{\def\PY@tc##1{\textcolor[rgb]{0.73,0.13,0.13}{##1}}}
\expandafter\def\csname PY@tok@mb\endcsname{\def\PY@tc##1{\textcolor[rgb]{0.40,0.40,0.40}{##1}}}
\expandafter\def\csname PY@tok@mf\endcsname{\def\PY@tc##1{\textcolor[rgb]{0.40,0.40,0.40}{##1}}}
\expandafter\def\csname PY@tok@mh\endcsname{\def\PY@tc##1{\textcolor[rgb]{0.40,0.40,0.40}{##1}}}
\expandafter\def\csname PY@tok@mi\endcsname{\def\PY@tc##1{\textcolor[rgb]{0.40,0.40,0.40}{##1}}}
\expandafter\def\csname PY@tok@il\endcsname{\def\PY@tc##1{\textcolor[rgb]{0.40,0.40,0.40}{##1}}}
\expandafter\def\csname PY@tok@mo\endcsname{\def\PY@tc##1{\textcolor[rgb]{0.40,0.40,0.40}{##1}}}
\expandafter\def\csname PY@tok@ch\endcsname{\let\PY@it=\textit\def\PY@tc##1{\textcolor[rgb]{0.25,0.50,0.50}{##1}}}
\expandafter\def\csname PY@tok@cm\endcsname{\let\PY@it=\textit\def\PY@tc##1{\textcolor[rgb]{0.25,0.50,0.50}{##1}}}
\expandafter\def\csname PY@tok@cpf\endcsname{\let\PY@it=\textit\def\PY@tc##1{\textcolor[rgb]{0.25,0.50,0.50}{##1}}}
\expandafter\def\csname PY@tok@c1\endcsname{\let\PY@it=\textit\def\PY@tc##1{\textcolor[rgb]{0.25,0.50,0.50}{##1}}}
\expandafter\def\csname PY@tok@cs\endcsname{\let\PY@it=\textit\def\PY@tc##1{\textcolor[rgb]{0.25,0.50,0.50}{##1}}}

\def\PYZbs{\char`\\}
\def\PYZus{\char`\_}
\def\PYZob{\char`\{}
\def\PYZcb{\char`\}}
\def\PYZca{\char`\^}
\def\PYZam{\char`\&}
\def\PYZlt{\char`\<}
\def\PYZgt{\char`\>}
\def\PYZsh{\char`\#}
\def\PYZpc{\char`\%}
\def\PYZdl{\char`\$}
\def\PYZhy{\char`\-}
\def\PYZsq{\char`\'}
\def\PYZdq{\char`\"}
\def\PYZti{\char`\~}
% for compatibility with earlier versions
\def\PYZat{@}
\def\PYZlb{[}
\def\PYZrb{]}
\makeatother


    % Exact colors from NB
    \definecolor{incolor}{rgb}{0.0, 0.0, 0.5}
    \definecolor{outcolor}{rgb}{0.545, 0.0, 0.0}



    
    % Prevent overflowing lines due to hard-to-break entities
    \sloppy 
    % Setup hyperref package
    \hypersetup{
      breaklinks=true,  % so long urls are correctly broken across lines
      colorlinks=true,
      urlcolor=urlcolor,
      linkcolor=linkcolor,
      citecolor=citecolor,
      }
    % Slightly bigger margins than the latex defaults
    
    \geometry{verbose,tmargin=1in,bmargin=1in,lmargin=1in,rmargin=1in}
    
    

    \begin{document}
    
    
    \maketitle
    
    

    
    \hypertarget{week-3-homework}{%
\section{Week 3 Homework}\label{week-3-homework}}

    \hypertarget{question-1}{%
\subsection{Question 1}\label{question-1}}

For each of the following cases, give an example demonstrating that
problem \(\mathbf{P}\) may have an optimal solution, and an example
demonstrating that \(\mathbf{P}\) may not have an optimal solution, or
argue that such an example does not exist.

\(P: min \ f(x) \ s.t. \ x \in X, X \subseteq \mathbf{R}^n\)

\hypertarget{a-the-function-f-is-discontinuous-and-the-set-x-is-compact.}{%
\paragraph{(a) The function f is discontinuous and the set X is
compact.}\label{a-the-function-f-is-discontinuous-and-the-set-x-is-compact.}}

Given that the set is compact, we know that it is bounded and closed.
Therefore, \(\mathbf{P}\) will not have a solution if the discontinuity
exists where the minimum would be, and \(\mathbf{P}\) will have a
solution if the discontinuity exists elsewhere.

\hypertarget{b-the-function-f-is-continuous-and-the-set-x-is-not-closed.}{%
\paragraph{(b) The function f is continuous and the set X is not
closed.}\label{b-the-function-f-is-continuous-and-the-set-x-is-not-closed.}}

If the set is not closed, then we may have an interval such as
\(-1 \lt x \le 1\). If \(f(x) = x\) then \(\mathbf{P}\) does not have a
solution since we can always choose a smaller \(x\). However, if
\(f(x) = x^2\) then \(\mathbf{P}\) does have a solution at \(x = 0\).

\hypertarget{c-the-function-f-is-convex-and-the-set-x-is-not-bounded.}{%
\paragraph{(c) The function f is convex and the set X is not
bounded.}\label{c-the-function-f-is-convex-and-the-set-x-is-not-bounded.}}

Given an unbounded set, \(\mathbf{P}\) will have a solution for some
convex functions but not others. For example, let \(X\) be a set that is
not bounded below. \(\mathbf{P}\) would have not a solution for
\(f(x) = x\). However, \(\mathbf{P}\) would have a solution for
\(f(x) = x^2\).

\hypertarget{d-the-function-f-is-convex-and-the-set-x-is-compact.}{%
\paragraph{(d) The function f is convex and the set X is
compact.}\label{d-the-function-f-is-convex-and-the-set-x-is-compact.}}

Given a compact set and a convex function, \(\mathbf{P}\) will always
have a soution. There are no examples of where \(\mathbf{P}\) does not
have a solution.

\hypertarget{e-the-function-f-is-linear-and-the-set-x-is-not-closed.}{%
\paragraph{(e) The function f is linear and the set X is not
closed.}\label{e-the-function-f-is-linear-and-the-set-x-is-not-closed.}}

If \(X\) is not closed, and \(f(x) = a*x + b\), then whether or not
\(\mathbf{P}\) has a solution depends on the constant \(a\) and the
boundry on which \(X\) is not closed. For example, if \(a \gt 0\), and
\(X\) includes the lower bound on its interval, then \(\mathbf{P}\) has
a solution. However, if \(a \gt 0\) and \(X\) does not include the lower
bound of the interval, then \(\mathbf{P}\) does not have a solution.

\hypertarget{f-the-function-f-is-linear-and-the-set-x-is-compact.}{%
\paragraph{(f) The function f is linear and the set X is
compact.}\label{f-the-function-f-is-linear-and-the-set-x-is-compact.}}

Given a compact set and a linear function, \(\mathbf{P}\) will always
have a soution. There are no examples of where \(\mathbf{P}\) does not
have a solution.

    \hypertarget{question-2}{%
\subsection{Question 2}\label{question-2}}

For each of the statements below, state whether it is true or is false.

\hypertarget{a-an-optimization-problem-with-a-discontinuous-objective-function-and-a-closed-and-bounded-feasible-region-can-never-have-an-optimal-solution.}{%
\paragraph{(a) An optimization problem with a discontinuous objective
function and a closed and bounded feasible region can never have an
optimal
solution.}\label{a-an-optimization-problem-with-a-discontinuous-objective-function-and-a-closed-and-bounded-feasible-region-can-never-have-an-optimal-solution.}}

This is false. The discontinuity doesn't necessarily impact the optimial
value.

\hypertarget{b-consider-the-optimization-problem-min-fx-s.t.-gx-le-0.-suppose-the-current-optimal-objective-value-is-v.-now-if-i-change-the-right-hand-side-of-the-constraint-to-1-and-resolve-the-problem-the-new-optimal-objective-value-will-be-less-than-or-equal-to-v.}{%
\paragraph{\texorpdfstring{(b) Consider the optimization problem
\(min \ f(x) \ s.t. \ g(x) \le 0\). Suppose the current optimal
objective value is \(v\). Now, if I change the right-hand-side of the
constraint to 1 and resolve the problem, the new optimal objective value
will be less than or equal to
\(v\).}{(b) Consider the optimization problem min \textbackslash{} f(x) \textbackslash{} s.t. \textbackslash{} g(x) \textbackslash{}le 0. Suppose the current optimal objective value is v. Now, if I change the right-hand-side of the constraint to 1 and resolve the problem, the new optimal objective value will be less than or equal to v.}}\label{b-consider-the-optimization-problem-min-fx-s.t.-gx-le-0.-suppose-the-current-optimal-objective-value-is-v.-now-if-i-change-the-right-hand-side-of-the-constraint-to-1-and-resolve-the-problem-the-new-optimal-objective-value-will-be-less-than-or-equal-to-v.}}

This is true. Increasing the upper bound on the constraint still
includes the original constraint, therefore the optimal value would only
stay the same or improve.

\hypertarget{c-consider-an-optimization-problem-p-min-fx-s.t.-x-in-x-where-x-is-a-non-empty-closed-convex-set.-suppose-that-the-problem-p-has-the-property-that-every-local-optimal-solution-is-also-globally-optimal-then-fx-must-be-a-convex-function.}{%
\paragraph{\texorpdfstring{(c) Consider an optimization problem
\((P) \ : \ min \ f(x) \ s.t. \ x \in X\), where \(X\) is a non-empty
closed convex set. Suppose that the problem \((P)\) has the property
that every local optimal solution is also globally optimal then \(f(x)\)
must be a convex
function.}{(c) Consider an optimization problem (P) \textbackslash{} : \textbackslash{} min \textbackslash{} f(x) \textbackslash{} s.t. \textbackslash{} x \textbackslash{}in X, where X is a non-empty closed convex set. Suppose that the problem (P) has the property that every local optimal solution is also globally optimal then f(x) must be a convex function.}}\label{c-consider-an-optimization-problem-p-min-fx-s.t.-x-in-x-where-x-is-a-non-empty-closed-convex-set.-suppose-that-the-problem-p-has-the-property-that-every-local-optimal-solution-is-also-globally-optimal-then-fx-must-be-a-convex-function.}}

This is false. Let \(f(x) = sin(x), 0 \le x \le 4\pi\). Each local
optimimum at a multiple of \(\frac{3\pi}{2}\) is also a global optimum,
but \(sin(x)\) is not convex.

\hypertarget{d-the-problem-p-min-x-y-subject-to-x2-y2-le-4-is-a-convex-optimization-problem.}{%
\paragraph{\texorpdfstring{(d) The problem
\((P): min \ x + y \ subject \ to \ x^2 + y^2 \le 4\) is a convex
optimization
problem.}{(d) The problem (P): min \textbackslash{} x + y \textbackslash{} subject \textbackslash{} to \textbackslash{} x\^{}2 + y\^{}2 \textbackslash{}le 4 is a convex optimization problem.}}\label{d-the-problem-p-min-x-y-subject-to-x2-y2-le-4-is-a-convex-optimization-problem.}}

This is true becuase the objective function is convex and the constraint
is convex.

\hypertarget{e-if-i-solve-an-optimization-problem-then-add-a-new-constraint-to-it-and-solve-it-again-the-solution-must-change.}{%
\paragraph{(e) If I solve an optimization problem, then add a new
constraint to it and solve it again, the solution must
change.}\label{e-if-i-solve-an-optimization-problem-then-add-a-new-constraint-to-it-and-solve-it-again-the-solution-must-change.}}

This is false. Adding a constraint doesn't necessarily change the
feasible region, and doesn't necessarily change the optimal solution.

\hypertarget{f-consider-the-following-optimization-problem-minfx2-s.t.-x-in-x-where-fx-is-a-nonconvex-function-and-x-is-a-non-empty-set.-suppose-at-a-feasible-solution-x-in-x-the-objective-value-is-0-then-x-must-be-an-optimal-solution.}{%
\paragraph{\texorpdfstring{(f) Consider the following optimization
problem: \(min[f(x)]^2 \ s.t. \ x \in X\) where \(f(x)\) is a nonconvex
function and \(X\) is a non-empty set. Suppose at a feasible solution
\(x^∗ \in X\) the objective value is 0, then \(x^∗\) must be an optimal
solution.}{(f) Consider the following optimization problem: min{[}f(x){]}\^{}2 \textbackslash{} s.t. \textbackslash{} x \textbackslash{}in X where f(x) is a nonconvex function and X is a non-empty set. Suppose at a feasible solution x\^{}∗ \textbackslash{}in X the objective value is 0, then x\^{}∗ must be an optimal solution.}}\label{f-consider-the-following-optimization-problem-minfx2-s.t.-x-in-x-where-fx-is-a-nonconvex-function-and-x-is-a-non-empty-set.-suppose-at-a-feasible-solution-x-in-x-the-objective-value-is-0-then-x-must-be-an-optimal-solution.}}

This is true. The minimum value of the square of any (real) valued
function is 0. The non-convexivity of \(f(x)\) is not a problem since we
are not concerned with whether or not \(x^*\) is a global optimum.

\hypertarget{g-if-i-maximize-a-univariate-convex-function-over-a-closed-interval-then-there-has-to-be-an-optimal-solution-which-is-one-of-the-end-points.}{%
\paragraph{(g) If I maximize a univariate convex function over a closed
interval then there has to be an optimal solution which is one of the
end
points.}\label{g-if-i-maximize-a-univariate-convex-function-over-a-closed-interval-then-there-has-to-be-an-optimal-solution-which-is-one-of-the-end-points.}}

This is true if the convex function is not a constant (e.g. \(f(x) = n\)
for some arbitrary number \(n\).


    % Add a bibliography block to the postdoc
    
    
    
    \end{document}
